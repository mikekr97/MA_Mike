\documentclass[11pt,a4paper,twoside]{book}
\input{header.sty}   % packages, layout and standard macros

\setcounter{secnumdepth}{3} % to show numbering for subsubsections


\begin{document}
\renewcommand\familydefault{\sfdefault} 
\pagenumbering{Alph}


\thispagestyle{empty}
\renewcommand{\baselinestretch}{1.5}\normalfont
\begin{center}
\setlength{\parindent}{0cm}
\bf\Large% 
Neural Causal Models with TRAM-DAGs: \\
A framework for modelling causal effects in a flexible and \\
interpretable way and for making subsequent causal queries.\\
\normalfont



\hrulefill

\vspace*{4cm}

\large
Master Thesis in Biostatistics (STA495) % or choose the next one
% Master Thesis in Mathematics (MAT491) 
\vspace*{12mm}

by

\vspace*{12mm}

Mike Kr{\"a}henb{\"u}hl\\
\small Matriculation number: 18-652-149\\
\normalfont
\vspace*{4cm}

supervised by

\vspace*{1cm}

Prof. Dr. Beate Sick \\
Prof. Dr. Oliver D{\"u}rr, HTWG Konstanz
\vfill

Zurich, July 2025
\end{center}
\renewcommand\familydefault{\rmdefault}%
\renewcommand{\baselinestretch}{1.0}\rm 
\setcounter{page}{0}
\newpage
\vspace*{12cm}~\thispagestyle{empty}\pagenumbering{Roman}
\newpage


\graphicspath{{./figure/}}
\DeclareGraphicsExtensions{.pdf,.png}
\setcounter{tocdepth}{1}

\thispagestyle{empty}
\begin{center}
%

	\vspace*{6cm}{\bfseries\Huge
	Causal Modeling with Neural Networks \\ [5mm]
and \\ [5mm]
Individualized Treatment Effect Estimation \\ [5mm]
	}
	
	% Functional Modeling with Neural Causal Models and Personalized Treatment Effect Estimation 
	% Modeling Functional Relationships in Causal Graphs and Estimating Individualized Interventions:\\[5mm]
	% Neural Causal Models (TRAM-DAGs) and Conditional Average Treatment Effects
	
  % \vspace*{6cm}{\bfseries\Huge
  % Neural Causal Models with TRAM-DAGs:\\[5mm]
  % Applied on real-world data \\[5mm]
  % and used for ITE estimation.
  % }

  \vfill
  \rm

  \LARGE
  Mike Kr{\"a}henb{\"u}hl\\[12mm]
  
  \normalsize
  Version \today
\end{center}
\newpage
\thispagestyle{empty}~
\newpage
\pagenumbering{roman}

\thispagestyle{plain}\markboth{Contents}{Contents}
\tableofcontents
\setkeys{Gin}{width=.8\textwidth}

\chapter*{Preface}
\addtocontents{toc}{\protect \vspace*{13.mm}}
\addcontentsline{toc}{chapter}{\bfseries{Preface}}
\thispagestyle{plain}\markboth{Preface}{Preface}



This thesis marks the final part of my Master of Science in Biostatistics at the University of Zurich. I wanted to work on a topic where I could apply my interest and deepen my knowledge in machine learning, especially in relation to causal questions.

The TRAM-DAG framework \citep{sick2025}, developed by my supervisors Prof. Dr. Beate Sick and Prof. Dr. Oliver D{\"u}rr, provided a perfect opportunity to do so. Our initial aim was to apply it to real-world data and potentially include semi-structured data. However, due to some surprising findings by \citet{chen2025}, our focus shifted towards the increasingly important topic of individualized treatment effect (ITE) estimation. Towards the end, we then bridged ITE estimation with the TRAM-DAG framework.

I want to thank my supervisors and all the people I had the chance to work and study with, as well as everyone who supported me on this journey.

% In the introduction, our aim is to give a summary of key concepts in causal inference and causal models. We also motivate the need for methods that allow drawing causal conclusions from observational data and introduce the proposed framework of TRAM-DAGs \citep{sick2025} as a tool that can be used for this purpose. We also want to highlight the importance of estimation of personalized treatment effects. \\ 
% 
% In the methods section, we give a detailed description of the TRAM-DAG framework, how it works, and for what kinds of causal queries the model can be used. Although individualized treatment effect (ITE) estimation is not a typical observational data problem, we also discuss important considerations and how TRAM-DAGs and other models can be applied in this context. \\ 
% 
% Then, we persent results of simulation studies that show the capabilities of TRAM-DAGs, the analysis of potential limitations in ITE estimation, and the TRAM-DAG applied for ITE estimation. Additionally, we perform ITE estimation on the International Stroke Trial (IST) with different causal ML methods including TRAM-DAGs.\\ 
% 
% In the final sections, we discuss the results and draw conclusions about the strengths and limitations of the framework and ITE esitimaiton in general, while also providing an outlook for future research. 


\bigskip

\begin{flushright}
  Mike Kr{\"a}henb{\"u}hl\\
  July 2025
\end{flushright}

\addtocontents{toc}{\protect \vspace*{10mm}}

\cleardoublepage



%%%%%%%%%%%%%%%%%%%%%%%%%%%%%%%%%%%%%%%%%%%%%%%%%%%%%%%%%%%%%%%%%%%%%% 


\chapter*{Abstract}
\addtocontents{toc}{\protect \vspace*{13.mm}}
\addcontentsline{toc}{chapter}{\bfseries{Abstract}}
\thispagestyle{plain}\markboth{Abstract}{Abstract}



This thesis explores the use of TRAM-DAGs, a flexible neural network-based framework for estimating complex causal relationships in known directed acyclic graphs (DAGs). TRAM-DAGs allow sampling from observational, interventional, and counterfactual distributions when the full DAG is known. We applied TRAM-DAGs to both simulated data and a real-world randomized controlled trial, with a focus on individualized treatment effect (ITE) estimation.

We show how TRAM-DAGs can be used with continuous and ordinal or categorical predictor variables, investigate how variable scaling affects interpretability, and demonstrate how interactions between variables can be modeled. A key part of this work involved applying different causal machine learning models -- including TRAM-DAGs, logistic regression, and random forests -- to estimate ITEs on the International Stroke Trial (IST) dataset. In line with findings by \citet{chen2025}, none of the models produced ITE estimates that generalized to the test data.

To explore possible reasons for this poor performance, we conducted simulation experiments under varying conditions. These revealed that, while models must be well calibrated and overfitting should be avoided, this alone may not suffice to guarantee valid ITE estimates. Weak treatment-covariate interaction effects and especially unmeasured effect modifiers were found to be critical challenges. When such variables are unobserved, the ignorability assumption alone may not ensure unbiased estimation -- an issue also highlighted by \citet{vegetabile2021}. These factors may help explain the limited model performance observed in the IST dataset.

We also applied TRAM-DAGs in randomized and confounded simulation settings with relatively complex DAGs and found that, when the full DAG was observed and interaction effects were present, TRAM-DAGs accurately recovered causal relationships and provided unbiased ITE estimates.

While promising, our work has limitations. The simulation scenarios may not fully capture real-world complexity, and evaluating ITE estimates on real data remains challenging since ground truth is unknown. The neural network-based TRAM-DAGs require training time and rely on modeling assumptions -- e.g., regarding the scale of conditional effects -- when parameter interpretability is desired.

TRAM-DAGs offer a customizable modeling framework that enables the specification of both flexibility and interpretability, making them suitable for real-world causal inference tasks. Future work could apply TRAM-DAGs to more diverse datasets, including semi-structured data, and further investigate ITE estimation in the presence of unmeasured effect modifiers.


% This thesis investigates the use of TRAM-DAGs \citep{sick2025} as a flexible approach for estimating structural equations in known directed acyclic graphs (DAGs). TRAM-DAGs offer several advantages: the model inherently knows when to control for covariates based on the DAG, is highly customizable in terms of flexibility and interpretability, and allows sampling from observational, interventional, and counterfactual distributions. We show how to incorporate ordinal predictors, model interactions, and examine how variable scaling affects interpretability.
% 
% A main focus was the estimation of individualized treatment effects (ITEs) using a variety of causal machine learning (ML) models. In simulation studies, we analyzed limitations in ITE estimation and found that unmeasured effect modifiers can severely impact estimation accuracy, and that the ignorability assumption alone may not ensure unbiased results -- a concern that has also been noted in prior research \citep{vegetabile2021}. The limitations found may also help explain the poor ITE estimation performance observed in the real-world application on the International Stroke Trial dataset \citep{chen2025}. We further demonstrated that TRAM-DAGs can be used for ITE estimation in relatively complex DAG structures, provided that the DAG is fully known and all variables are observed.
% 
% While promising, TRAM-DAGs require training time due to their reliance on neural networks and, when aiming for interpretability, certain assumptions about the model structure. Future work could explore applications to more real-world data, potentially including semi-structured inputs, and further investigate ITE estimation in the presence of unmeasured interactions.
% 
% This thesis contributes to the field of causal inference under observational data and to the estimation of personalized treatment effects using causal ML models.




% This thesis investigates the use of TRAM-DAGs as a flexible approach for estimating structural equations in known DAGs. TRAM-DAGs offer several advantages: the model knows when to adjust based on the DAG, avoids incorrect covariate adjustment, and allows sampling from observational, interventional, and counterfactual distributions. Their ability to combine interpretability with flexibility makes them well suited for practical use.
% We show how to incorporate ordinal predictors, model interactions, and how variable scaling affects interpretability. A main focus was the estimation of individualized treatment effects (ITEs), using a variety of causal machine learning (ML) models.
% In simulation studies, we analyzed limitations in ITE estimation where we found that However, we also found that unmeasured effect modifiers can severely impact ITE estimation, and that the ignorability assumption alone may not ensure unbiased results. We demonstrated that TRAM-DAGs can also be applied for ITE estimation in relatively complex DAG structures, when the DAG is fully known and all variables are observed.
% 
% The found limitations may also reflected reasons for poor ITE estimation performance in the real-world application on the International Stroke Trial.
% 
% While promising, TRAM-DAGs rely on neural networks and require training time and assumptions about model structure. Future work could apply them to more real-world data, possibly including semi-structured inputs, and further investigate ITE estimation with unmeasured interactions.
% 
% This thesis contributes to the field of causal inference under observational data and towards personalized treatment effect estimation using causal machine learning models.

\bigskip

\noindent\textbf{Keywords:} TRAM-DAGs, neural causal model, individualized treatment effect, structural causal model, counterfactuals, transformation model, observational data, heterogeneous treatment effect, conditional average treatment effect




\addtocontents{toc}{\protect \vspace*{10mm}}

\cleardoublepage

%%%%%%%%%%%%%%%%%%%%%%%%%%%%%%%%%%%%%%%%%%%%%%%%%%%%%%%%%%%%%%%%%%%%%% 

\pagenumbering{arabic}


%%%%%%%%%%%%%%%%%%%%%%%%%%%%%%%%%%%%%%%%%%%%%%%%%%%%%%%%%%%%%%%%%%%%%% 
%%%%%%%%%%%%%%%%%%%%%%%%%%%%%%%%%%%%%%%%%%%%%%%%%%%%%%%%%%%%%%%%%%%%%%

% LaTeX file for Chapter 01



%  possible title: Causal inference for non-randomized data and heterogeneous treatment effect estimation


\chapter{Introduction}

\section{Motivation}

% Why, what and how\dots

The most important questions in research are mostly not associational, but causal \citep{pearl2009}. They concern the effects of interventions -- such as the impact of a treatment -- or seek explanations for observed outcomes, such as identifying which disease caused certain symptoms. They also include hypothetical scenarios; for example: what would the GDP have been if interest rates had increased by 25 instead of 75 basis points? Answering such questions requires causal reasoning and demands an understanding of the underlying data-generating process. Purely associational approaches are typically not sufficient to draw valid causal conclusions.

% https://ascpt.onlinelibrary.wiley.com/doi/epdf/10.1002/cpt.3159
% One of the main reasons for this is thatML methods are particularly good at learning about the status quofrom existing data to ultimately make outcome predictions thatare exactly in line with the current distribution of data character-istics, but usually not designed for tasks that involve the need toreason about interventions on the data generating distributionsleading to counterfactual scenarios,


The gold standard for estimating the causal effect of an intervention on an outcome is the randomized controlled trial (RCT) \citep{hariton2018}. In this prospective study design, participants are randomly assigned to either the treatment or control group. Randomization aims to eliminate the influence of potential confounding variables, ensuring that treatment groups are balanced with respect to baseline characteristics. This allows for an unbiased estimation of the causal effect. Despite their strengths, RCTs have several limitations. They are often expensive and time-consuming to plan and execute. Moreover, the results may not generalize well to the population of interest, as individuals who volunteer or are accepted for trials are not always representative of the target group. Additionally, RCTs typically estimate an average treatment effect (ATE) on a sample, which is the difference in mean outcomes between treatment arms \citep{nichols2007}. However, individual patients may respond differently to the treatment, depending on their unique characteristics. In the context of personalized medicine, it is therefore crucial to have an estimate of treatment effects at the individual level. Another central limitation of RCTs is that in many scenarios they can simply not be conducted due to ethical or practical reasons. For example, an RCT is only ethical in the case of clinical equipoise, which means that there is uncertainty about the (superiority) of one of the two treatment arms \citep{freedman1987}. It is not acceptable to treat one group with the assumed inferior treatment. The same is true for obviously harmful interventions, like smoking or drinking alcohol. In these cases, it is not possible to conduct an RCT to estimate the causal effect of smoking on lung cancer. 

For these reasons, much of research aims to make causal inference from observational data, using non-experimental or quasi-experimental designs. Unlike RCTs, these settings do not involve randomization to treatment, which introduces challenges due to confounding. Methods for causal inference from observational data aim to correctly control for such confounders to enable valid causal conclusions. Recently, \citet{sick2025} proposed the TRAM-DAGs framework, which estimates the functional form of causal relationships in a known causal graph based on observational or RCT data and make subsequent causal queries. In this thesis we further analyze and apply this method.

As mentioned earlier, an application where causal inference is of particular importance is the estimation of personalized treatment effects. In personalized medicine, this is referred to as the individualized treatment effect (ITE) or conditional average treatment effect (CATE), while in business and marketing contexts, the term uplift modeling is often used \citep{gutierrez2017, zhao2020}. These concepts refer to the difference in potential outcomes under different treatments, assessed at the level of individuals or subgroups. Such estimates are critical in settings where treatment responses vary significantly between individuals. For clinical decision-making, tailoring therapies to individual characteristics can lead to more effective and efficient care. The importance of estimating individual-level effects is not limited to medicine. It also is of high interest in marketing, where campaigns can be precisely targeted to maximize impact and minimize adverse responses. Consider, for instance, the decision of whether to send a push notification (treatment) to a customer. Some customers might be persuadables, who will respond positively only if treated. Others, in contrast, might have responded positively without the intervention but are negatively affected by it -- for example, a customer who is reminded of a forgotten subscription and, as a result, decides to cancel it. In this context, identifying persuadables is valuable, while treating the latter may be counterproductive. This illustrates the need to understand treatment effects at a granular level to guide individualized decisions.
Various methods have been proposed to estimate individualized treatment effects, yet this task remains challenging. The fundamental problem is that only one of the two potential outcomes can ever be observed for any given individual \citep{holland1986}, making the estimation of treatment effects inherently more difficult than standard predictive modeling.

% \citep{gutierrez2017, zhao2020}








\section{Key concepts of causal inference}

Causal relationships can be represented by a directed acyclic graph (DAG), as shown in Figure~\ref{fig:pearl_levels}(a). The variables, or nodes, are connected by directed edges, which represent causal dependencies.

% Visual examples of these levels are presented in Figure~\ref{fig:pearl_levels}(a)--(c). redundant, later i refer to it again
Questions in causal inference are typically classified into one of the three levels of Pearl's hierarchy of causation \citep{pearl_book2009}. Level 1 corresponds to observational queries, expressed as conditional probabilities $\text{P}(Y \mid X)$, which can be answered directly from the joint distribution $\text{P}(Y \cap X)$. Level 2 involves interventional queries, such as $\text{P}(Y \mid \text{do}(X = \alpha))$, which describe the probability when actively setting a variable $X$ to a particular value. Unlike observational queries, answering interventional questions requires knowledge of the underlying causal structure. Level 3 addresses counterfactual reasoning, which poses the greatest challenge. These are hypothetical what-if questions that require reasoning about outcomes under alternative realities. For example, if a patient received a treatment and died, the factual outcome is death under the received treatment. The counterfactual would be the outcome that would have occurred had the patient received a different treatment.

 
Some statisticians argue that counterfactuals -- being unobservable and untestable -- are of limited scientific value and may be regarded as metaphysical \citep{dawid2000}. Nevertheless, there are important practical questions that require the analysis of such counterfactuals.



% include image /img/pearl_levels.png
\begin{figure}[H]
\centering
\includegraphics[width=1\textwidth]{img/pearl_levels.png}
\caption{Illustration of the three levels of Pearl's hierarchy of causation. (a) Directed acyclic graph (DAG) for observational data. (b) DAG when making a do-intervention by fixing the variable $E$ at a certain value. (c) Observed factual outcome and the corresponding counterfactual query.}
\label{fig:pearl_levels}
\end{figure}


To illustrate Pearl's three levels of causality, we consider a simplified example involving the exposure Exercise ($E$), the outcome Heart Disease ($Y$), the confounder Age ($X_1$) and the additional covariate Smoking ($X_2$). I assume that exercise reduces the risk of heart disease, but both variables are also influenced by age. Figure~\ref{fig:pearl_levels}(a)--(c) illustrates the corresponding scenarios.





\textbf{Level 1: Observational ("seeing"):}  
We observe the joint distribution of variables without intervention.  
Example: What is the probability of heart disease given that a person exercises?  
\[
P(Y = 1 \mid E = 1)
\]
This can be estimated directly from data by conditioning on $E = 1$ and computing the frequency of $Y = 1$. However, such an estimate does not account for confounding variables like age.



\textbf{Level 2: Interventional ("doing"):}  
We consider the effect of actively intervening in the system.  
Example: What is the probability of heart disease if everyone were made to exercise, regardless of age or smoking status?  
\[
P(Y = 1 \mid \text{do}(E = 1))
\]
Answering this requires assumptions about the underlying causal structure.



\textbf{Level 3: Counterfactual ("imagining"):}  
We ask what would have happened under different circumstances -- that is, we imagine an alternative scenario for the same individual.  
Example: For a person who does not exercise and has heart disease, would they still have had heart disease if they had exercised?  
\[
P(Y_{E=1} \mid E = 0, Y = 1)
\]

Here, $Y_{E=1}$ represents the counterfactual outcome under positive exposure. Counterfactual queries cannot be answered from observational data alone; they require a structural framework that explicitly models the data-generating process. 


% With our new framework, we can answer all three types of questions, with a small exception for Counterfactuals in the non-continuous case.



% the following all from pearl book p. 27
For this, the concept of DAGs can be extended to structural causal model (SCM). A set of structural equations of the form $X_i = f_i(\text{pa}(X_i), Z_i), \quad i = 1, \dots, n$ build a structural causal model \citep{pearl_book2009}.  $\text{pa}(X_i)$ denotes the direct causal parents of $X_i$, and $Z_i$ is an exogenous noise variable. These exogenous variables capture latent factors that influence $X_i$ but are not explicitly modeled. By convention, the $Z_i$ are assumed to be mutually independent.

Each function $f_i$ -- which may be nonlinear -- defines how the value of $X_i$ is generated from its parents and the corresponding noise term. A source node $X_j$ without any parents is modeled as $X_j = f_j(Z_j)$. Once all structural equations and noise variables are specified, the model is fully deterministic in the sense that each variable is a fixed function of its parents and its own exogenous noise. The randomness in the system arises entirely from these independent noise terms, which encode unobserved factors. This functional representation makes it possible to compute interventional distributions and evaluate counterfactual outcomes. These aspects are discussed in detail in Section~\ref{methods:sampling}.


In this thesis, we do not focus on discovering the underlying causal graph. Such a structure may be obtained through structure learning algorithms or determined from expert knowledge. Instead, we assume the graph is known and concentrate on estimating the functional form of the relationships between variables -- that is, the structural equations that define the SCM.

Various approaches exist for estimating the functions $f_i$ that constitute an SCM, depending on the assumptions made about the data and the model class. These methods are discussed in the next section.




% \[
% X_i = \beta_0 + \beta_1 X_{pa(x_i)} + Z_i
% \]
%
% \[
%   X \sim \mathcal{N}(\mu = pa(X)\beta,\,\sigma^{2})\,.
% \]


A simple approach to modeling the structural equations is linear regression, which assumes Gaussian error terms $Z_i$ and linear functional forms $f_i$. Classical statistical methods of this kind are typically well-defined, computationally efficient, and offer interpretable parameters. However, they rely on strong assumptions about the underlying data-generating mechanism -- such as linearity and homoscedasticity -- which may not hold in practice. Violations of these assumptions can lead to biased or misleading results.

Alternatively, more flexible approaches based on neural networks have gained popularity for estimating structural equations. These models are capable of approximating complex, nonlinear relationships and capturing complicated interactions between variables with minimal bias. Their flexibility, however, often comes at the cost of reduced interpretability and, in some cases, limited applicability to non-continuous or mixed data types. \citet{poinsot2024} provided an overview of deep structural causal models and their use in counterfactual inference.

The TRAM-DAG framework proposed by \citet{sick2025} builds a bridge between these classical and neural-network-based modeling approaches, by combining interpretable transformation models with the flexibility of neural networks. At its core, the structural equations are modeled using transformation models \citep{hothorn2014}, a flexible class of distributional regression methods. These models were subsequently extended to deep transformation models (Deep TRAMs) by \citet{sick2020}, enabling the use of neural networks to parameterize conditional distributions in a customizable way. In the TRAM-DAG framework, these deep TRAMs are applied according to a known causal graph, allowing the model to be fitted to observational data and used to answer causal queries across all three levels of Pearl's hierarchy. The framework is introduced in more detail in Section~\ref{sec:tram_dags}.





\section{Goals and contributions} \label{sec:goals_contributions}

This thesis contributes to the further exploration of the TRAM-DAG framework and to adressing challenges in the estimation of personalized treatment effects.

The first part of this thesis focuses on a systematic analysis and extension of TRAM-DAGs. This includes applying the model across a variety of settings, such as different data types, model complexities, and neural network configurations (e.g., activation functions, batch normalization, dropout). Most analyses are conducted on simulated data to know the underlying data-generating process, but the model is also applied to real-world data to demonstrate its practical utility.

The second focus of the thesis is on the estimation of personalized treatment effects. Recent work by \citet{chen2025} showed that most causal machine learning models trained on RCT data failed to generalize when evaluated out of sample. In this thesis, we replicate some of their work by applying various models, including TRAM-DAGs, to the same data and analyzing whether we come to a similar conclusion. We further investigate why individualized treatment effect (ITE) estimation can fail in such settings, and under which conditions reliable estimates can be obtained. In addition, we demonstrate that TRAM-DAGs can be effectively used to estimate ITEs also in non-randomized observational settings, provided that the causal graph is known and fully observed. In doing so, we explore the potential of TRAM-DAGs as a framework for answering complex causal questions across different levels of Pearl's hierarchy.



Formally, we aim to answer following research questions in this thesis:

\begin{itemize}
    \item How can TRAM-DAGs be applied under different scenarios such as ordinal predictors, scaled vs. raw variables or allowing for interactions between variables?
    \item Do we obtain similar results when estimating ITEs on a real-world RCT dataset, as reported by \citet{chen2025}?
    \item What are possible reasons for the failure of ITE estimation in some cases when causal machine learning models are validated out of sample?
    \item How can TRAM-DAGs be used to estimate ITEs in both randomized controlled trials and observational settings involving confounding and mediating variables?
\end{itemize}



With this work, we aim to contribute to the important and evolving field of causal inference in observational settings and to the challenging task of estimating individualized treatment effects.


%%%%%%%%%%%%%%%%%%%%%%%%%%%%%%%%%%%%%%%%%%%%%%%%%%%%%%%%%%%%%%%%%%%%%% 
%%%%%%%%%%%%%%%%%%%%%%%%%%%%%%%%%%%%%%%%%%%%%%%%%%%%%%%%%%%%%%%%%%%%%%



% LaTeX file for Chapter 02


\chapter{Methods} 

In this section I will explain the necessary background needed to understand the TRAM-DAGs. Once the framework of tram dags is explained, I will present how the experiments of the simulation, the application on real data and the ITE estimation are conducted.


The goal of TRAM-DAGs is to estimate the structural equations according to the causal order in a given DAG in a flexible and possibly still interpretable way in order to sample observational and interventional distributions and to make counterfactual statements. The estimation requires data and a DAG that describes the causal structure. It must be assumed that there are no hidden confounders. TRAM-DAGs estimate for each variable $X_i$ a transformation function $Z_i = h_i(X_i \mid pa(X_i))$, where $Z_i$ is the noise value and $pa(X_i)$ are the causal parents of $X_i$. The important part here is that we can rearrange this equation to $X_i = h_i^{-1}(Z_i \mid pa(x_i))$ to get to the structural equation. The transformation functions $h$ are monotonically increasing functions that are a representation of the conditional distribution of $X_i$ on a latent scale. They are based on the idea of transformation models as introduced by \citet{hothorn2014} but were extended to deep trams by \citet{sick2020}. In the following sections I review the most important ideas of these methods as they are the essential components of TRAM-DAGs.

\section{Transformation Models}


Transformation models are a flexible distributional regression method for various data types. They can be for example specified as ordinary linear regression, logistic regression or proportional odds logistic regression. But Transformation models further allow to model conditional outcome distributions that do not even need to belong to a known distribution family of distributions by model it in parts flexibly. This reduces the strength of the assumptions that have to be made.

The basic form of transformation models can be described by 

\begin{equation}
F(y|\mathbf{x}) = F_Z(h(y \mid \mathbf{x}) =  F_Z(h_I(y) - \mathbf{x}^\top \boldsymbol{\beta})
\end{equation}

, where $F(y|\mathbf{x})$ is the conditional cumulative distribution function of the outcome variable $Y$ given the predictors $\mathbf{x}$. $h(y \mid \mathbf{x})$ is a transformation function that maps the outcome variable $y$ onto the latent scale of $Z$. $F_Z$ is the cumulative distribution function of a latent variable $Z$, the so-called inverse-link function that maps $h(y \mid \mathbf{x})$ to probabilities. In this basic version, the transformation function can be split into an intercept part $h_I(y)$ and a linear shift part $\mathbf{x}^\top \boldsymbol{\beta}$, where the vector $\mathbf{x}$ are the predictors and $\boldsymbol{\beta}$ are the corresponding coefficients.

If the latent distribution $Z$ is chosen to be the standard logistic distribution, then the coefficient $\beta_i$ can be interpreted as log-odds ratios when increasing the predictor $x_i$ by one unit, holding all other predictors unchanged. This means that an increase of one unit in the predictor $x_i$ leads to an increase of the log-odds of the outcome $Y$ by $\boldsymbol{\beta}$. The additive shift of the transformation function means a linear shift on the latent scale (herer log-odds). The following transformation to probabilities by $F_Z$ potentially leads to a non-linear change in the conditional outcome distribution on the original scale. This means not only is the distribution shifted, also its shape can change to some degree based on the covariates. More details about the choice of the latent distribution and the interpretation of the coefficients are provided in the appendix XXX. 


For a continuous outcome $Y$ the intercept $h_I$ is represented by a bernstein polynomial, which is a flexible and monotonically increasing function

\begin{equation}
h_I(y) = \frac{1}{M + 1} \sum_{k=0}^{M} \vartheta_k \, \text{B}_{k, M}(y)
\end{equation}

, where $\vartheta_k$ are the coefficients of the bernstein polynomial and $\text{B}_{k, M}(y)$ are the Bernstein basis polynomials. More details about the technical implementation of the bernstein polynomial in the context of TRAM-DAGs is given in the appendix XXX.

For a discrete outcome $Y$ the intercept $h_I$ is represented by cut-points, which are the thresholds that separate the different levels of the outcome. For example, for a binary outcome $Y$ there is one cut-point and for an ordinal outcome with $K$ levels there are $K-1$ cut-points. The transformation model is given by

\begin{equation}
P(Y \leq y_k \mid \mathbf{X} = \mathbf{x}) = F_Z(\vartheta_k + \mathbf{x}^\top \boldsymbol{\beta}), \quad k = 1, 2, \ldots, K - 1
\end{equation}


A visual representation for a continuous and discrete (ordinal) outcome is provided in Figure~\ref{fig:tram_cont_ord}.


% include image /img/tram_cont_ord.png
\begin{figure}[H]
\centering
\includegraphics[width=1\textwidth]{img/tram_cont_ord.png}
\caption{\textbf{Left:} Example of a transformation model for a continuous outcome $Y$ with a smooth transformation function. \textbf{Right:} Example of a transformation model for an ordinal outcome $Y$ with 5 levels. The transformation function consists of cut-points that separate the probabilities for the levels of the outcome.
In both cases the latent distribution $Z$ is the standard logistic and the predictors $\mathbf{x}$ induce a linear (vertical) shift of the transformation function.}
\label{fig:tram_cont_ord}
\end{figure}


For the remainder of this thesis, I rely on the idea of these transformation models to model the conditional distribution functions represented by the transformation functions of the respective variables. The standard logistic distribution is used as $F_Z$, which results in a logistic transformation model.


\section{Deep TRAMs}

The transformation models as discussed before were extended to deep TRAMs using neural networks. The goal is to get a parametrized transformation function by training a modular neural network and thereby minimizing the NLL. This minimization is done with Deep learning optimization methods.

\begin{figure}[H]
\centering
\includegraphics[width=0.9\textwidth]{img/deep_tram.png}
\caption{Modular deep transformation model. The transformation function $h(y \mid \mathbf{x})$ is constructed by the outputs of three neural networks.}
\label{fig:deep_tram}
\end{figure}

So on the slides you can see again the transformation function for the outcome Y that depends on the covariates X. Here, XL should stand for the predictors that shift the transformation function in a linear manner (as in the examples on the previous slides) and XC stands for Predictors that have a non linear influence. And the cool thing here is that these Complex Predictors could also include something like an Image.

The first part of the transformation function is a Simple intercept which is responsible for the baseline shape of the distribution. This intercept is constructed by a neural network, that takes no predictors as input and just outputs the parameters for the Bernstein polynomial or for the cut points in the discrete case. If we wanted to allow for more flexibility, we could give a predictor as input instead of just a constant, but in this presentation I will only show the simple case.

Next there is a Linear Shift which is basically a linear combination of the linear predictors. We can obtain this by creating another neural network with no hidden layers taking only the predictors as input and producing the linear shift as output.

Finally there is the complex shift which we obtain by giving the complex predictors into a neural network with some hidden layers and getting a single number as output.

The Neural Network training happens iteratively by starting with a (random) parameter configuration and then using the outputted intercepts and shift to calculate the loss, which is the negative log likelihood of our transformation model. To improve the loss, the parameters are then slightly adjusted by the adam optimizer. Then the process is repeated until the parameter estimates converge. So this means we are optimizing the negative log likelihood of the logistic transformation model, which represents the conditional distribution function of our outcome variable. 

The nice thing of these deep transformation models is that we can specify, whether we want a linear shift bx or a complex shift b(x) or even a complex intercept and thereby controlling the flexibility.


\textbf{TRAM-DAGs}

And here you can see how this looks in our case, where we apply these deep transformation models in a causal setting. So we assume a pre-specified DAG which defines the Causal dependence. And then we model each node by a transformation model that is conditional on its parents.

In this Example dag we have 3 variables:

X1 is continuous and a source node, meaning it has no parents that it depends on, hence the transformation function h only represents the intercept.

Then X2 depends on X1, so this means that the transformation function changes with X1. In what way the transformation function is allowed to change, has to be specified. This could either be a linear or complex shift or even a complex intercept.

And finally there is the ordinal variable X3 which has 4 levels. The Transformation function consists of 3 cut points that depend on X1 and X2. These cut-points represent the Probabilities of the 4 levels of X3.

Choice of link function does not matter if fully flexible (CI) but it puts some assumptions if not fully flexible. Same as in the sense of the SCM where the choice of the Noise distribution can matter (depending on how flexible the equations are).


Ok so lets make a simulation example and put this all together.

First of all we have observational data that follows a pre-defined DAG without hidden confounders. In practice such a dag can be defined by expert knowledge or by some sort of structure finding algorithm.

Then we want to estimate the conditional distribution function of each variable so that we can sample from the distributions and make causal queries.

So in this example we assume the same dag as on the previous slide with the 3 variables. X1 and X2 are continuous and X3 is ordinal. Now we also specify how these variables are related. So X1 is the source node. X2 depends on X1 through a linear shift and X3 depends on X1 by a linear shift and on X2 by a complex shift.


This model structure can also be presented by an adjacency matrix where the rows indicate the source of effect and the columns are the target of the effect. In our algorithm we use this Meta-adjacency matrix to control the information flow.





\textbf{Construct Modular Neural network}

Now we set up a modular neural network which produces the components of the transformation functions as outputs. The Adjacency Matrix thereby controls the information flow. Here you also can see the number of parameters. In total there are 281 Parameters in this model, but not all of them are used I will not get into the details here. On the right side you can also see the neural network that produces the complex shift from X2 to X3. We chose to model it by 4 hidden layers with 2 nodes each. That should allow for high enough flexibility.

Finally the outputs of the different Neural network parts are combined to the transformation functions for each node, from which we then derive the negative log likelihood.

Optimizers and hyperparameters etc

Adam, Batchnormalization, activation functions (relu, sigmoid)
Impact of Scaling (maybe in Appendix)
Neural network works best if inputs are scaled. Proof that we can do that, it just changes the interpretation. For structure finding algorithms, this might be problematic, because increasing variance along the causal order would be destroyed. (why, how, interpretation change etc. check meeting notes 22.04.2025)

Different Intercepts, and Shifts

and show, describe how the transformation function and hence the conditional distribution will change in each scenario. Also show in detail how the neural networks would be set up, how the information flow is controlled and what kind of outputs are produced and how they further have to be transformed.

Describe interpretation quickly and refer to formal proof in the Appendix.
%  for interpretation see pearl book 2009 p. 366. the key is to say leaving all other variables "untouched" and not "constant". he also talks about the connection to the do-operator.

Fitting Betas Interpretable

The two parameters for our linear shift terms are plotted here. We can see that they converge quickly to the same values as we used in the DGP. We can interpret these parameters as log-odds ratios if changing the value of the parent by one unit.

Intercepts

Show the Discrete case with just cutpints (only K-1 parameters of outputs are used)
Show the continuous case where the outputs are transformed to monotonically increasing betas for the bernstein polynomial. Also describe Bernstein polynomial construction in detail with scaling and linear extrapolation.

Here I plotted the intercepts of the 3 transformation functions. They also resemble the DGP very nicely.


Linear and complex shifts

Here in the first two plots we can see the linear shifts. And in the right plot we have the complex shift of X2 on X3. The estimated shifts match quite well with the DGP.

Complex shift (Interaction example) to show what is also possible

Here I just want to make a short input from another example. So there the true model was that of a logistic regression with the binary outcome Y and 3 predictors. The binary treatment T and the two continuous predictors X1 and X2. There was also an interaction effect assumed between treatment and X1. So this basically means that the effect of X1 on the outcome is different for the two treatment groups.

And here we can show that our TRAM-DAG specified by a complex shift of T and X1 can also capture this interaction effect quite well.


Loss function

And this is how the negative log likelihood looks like for a continuous outcome. It is derived from the CDF based on the logistic transformation model. A special thing here is that the outcome variable has to be scaled to the range between 0 and 1 first, and this scaling also has to be considered when calculating the NLL. But I will not go through this now.
For final Loss, the individual losses of the nodes are added together. (only in R framework, in Python they are fitted individually?)


NN Training

Now we have everything in place to train the neural network. Here I run the model for 400 Epochs (which means that the model has seen each sample 400 times). On the right you can see the loss of the training set and also the validation loss as comparison. The NLL quickly dropped to around 1.1 and then didnt change much anymore, which indicates that the parameter estimates have probably converged to a good state.


\textbf{discrete predictors}

\textbf{sampling from the tram dag}
Each variable Xi in the DAG can be described by a structural equation Xi = f(Zi, pa(Xi)). For a continous outvome x1, in TRAM-DAGs this structural equation is the inverse of the conditional transformation function Xi = h-1(Zi | pa(Xi)). for a discrete outcome it is defined as... show sampling.

Refer back to the SCM, that we basically can obtain the structural equations from our model. Okay so now we have estimates for the conditional transformation functions of our 3 variables. To generate a sample for a node, we first sample a random value from the latent distribution. In our case from the standard logistic. We denote this sample as Z.

Next we want to determine the value of the Node X. If X is continuous, we can apply the inverse of the transformation function evaluated at Z to find X. If X is ordinal, we just select the corresponding category that belongs to the next bigger cut-point.

If we want to make a do-intervention, we just fix a node at the desired value.


\textbf{counterfactuals}

Describe how to do it, limitations etc.

see pearl book causality: 1.4.4 Counterfactuals in Functional Models


\textbf{ITE how it is applied in our model?}

RCTs only measure the average treatment effect. There will be patients who respond better or worse to the treatment because patient specific characteristcs. In personalized medicine however, the aim is to find the optimal treatment for a specific individual. Such a measure that can help in decision making is the ITE.

Rubins potential outcomes framework.


Maybe it is the methods section. Here however, we give a couple hints.
Note that you can wisely use \rr{preamble}-chunks. Minimal, is likely: 

\bigskip

\hrule
\begin{knitrout}
\definecolor{shadecolor}{rgb}{0.969, 0.969, 0.969}\color{fgcolor}\begin{kframe}
\begin{verbatim}
library(knitr) 
opts_chunk$set( 
    fig.path='figure/ch02_fig',    
    self.contained=FALSE,
    cache=TRUE
) 
\end{verbatim}
\end{kframe}
\end{knitrout}
\hrule

\bigskip

Defining figure options is very helpful:

 
\bigskip


\hrule
\begin{knitrout}
\definecolor{shadecolor}{rgb}{0.969, 0.969, 0.969}\color{fgcolor}\begin{kframe}
\begin{verbatim}
library(knitr)
opts_chunk$set(fig.path='figure/ch02_fig',
               echo=TRUE, message=FALSE,
               fig.width=8, fig.height=2.5,  
               out.width='\\textwidth-3cm',
               message=FALSE, fig.align='center',
               background="gray98", tidy=FALSE, #tidy.opts=list(width.cutoff=60),
               cache=TRUE
) 
options(width=74)
\end{verbatim}
\end{kframe}
\end{knitrout}
\hrule

\bigskip 

This options are best placed in the main document at the beginning. Otherwise a \verb+cache=FALSE+ as knitr option is necessary to overrule a possible  \verb+cache=TRUE+ flag. 

\bigskip 

Notice how in Figure~\ref{f02:1} everything is properly scaled.   

\begin{figure}
\begin{knitrout}
\definecolor{shadecolor}{rgb}{0.98, 0.98, 0.98}\color{fgcolor}

{\centering \includegraphics[width=\textwidth-3cm]{figure/ch02_figunnamed-chunk-3-1} 

}


\end{knitrout}
  \caption{Test figure to illustrate figure options used by knitr.}
  \label{f02:1}
\end{figure}


\section{Citations}

Recall the difference between \verb+\citet{}+ (e.g., \citet{Chu:Geor:99}), \verb+\citep{}+ (e.g., \citep{Chu:Geor:99}) and \verb+\citealp{}+ (e.g., \citealp{Chu:Geor:99}).
For simplicity, we include here all references in the file \verb+biblio.bib+ with the command \verb+\nocite{*}+.\nocite{*}



%%%%%%%%%%%%%%%%%%%%%%%%%%%%%%%%%%%%%%%%%%%%%%%%%%%%%%%%%%%%%%%%%%%%%%
%%%%%%%%%%%%%%%%%%%%%%%%%%%%%%%%%%%%%%%%%%%%%%%%%%%%%%%%%%%%%%%%%%%%%%


% \input{chapter_Exp1}

%%%%%%%%%%%%%%%%%%%%%%%%%%%%%%%%%%%%%%%%%%%%%%%%%%%%%%%%%%%%%%%%%%%%%%
%%%%%%%%%%%%%%%%%%%%%%%%%%%%%%%%%%%%%%%%%%%%%%%%%%%%%%%%%%%%%%%%%%%%%%

% 

% LaTeX file for Chapter Exp2






\chapter{Experiment 2: ITE on International Stroke Trial (IST)}





\section{Motivation}


% describe the data of stroke trial https://pubmed.ncbi.nlm.nih.gov/9174558/
% Results on IST trial with the interpretation in the discussion part.

%  here the authors made the IST database available and described the trial, we downloaded the CSV
% https://trialsjournal.biomedcentral.com/articles/10.1186/1745-6215-12-101 


% Results: The IST dataset includes data on 19 435 patients with acute stroke, with 99% complete follow-up. Over 26.4% patients were aged over 80 years at study entry. Background stroke care was limited and none of the patients received thrombolytic therapy.
% 
% 
\citet{chen2025} evaluated multiple causal ML methods on the International Stroke Trial (IST), to estimate the individualized treatment effects. They demonstrated that none of the applied ML methods generalized well, as performance on the test data differed significantly from the training data on the chosen evaluation metrics.
In this experiment, we replicate the analysis on the same data by applying three causal ML methods for ITE estimation, to investigate whether we obtain similar results as the authors.


\section{Setup} \label{sec:methods_experiment2}



\textbf{Data:} The International Stroke Trial was a large, randomized controlled trial conducted in the 1990s to assess the efficacy and safety of early antithrombotic treatment in patients with acute ischemic stroke \citep{IST1997}. Using a 2x2 factorial design, 19,435 patients across 36 countries were randomized within 48 hours of symptom onset to receive aspirin, subcutaneous heparin, both, or neither. Patients allocated to aspirin (300 mg daily for 14 days) had a 6-month death or dependency rate of 62.2\%, compared to 63.5\% in the control group not receiving aspirin, corresponding to a statistically significant absolute risk reduction after adjustment for baseline prognosis (1.4\%, p = 0.03). The authors stated that there was no interaction between aspirin and heparin in the main outcomes. In this thesis, we focus exclusively on the aspirin vs. no aspirin comparison and the outcome of death or dependency at 6 months after stroke.

The dataset used in this experiment was made publicly available by \citet{sandercock2011} and contains individual-level data, including baseline covariates assessed at randomization, treatment allocation, and 6-month outcomes, with a follow-up rate of 99\%.

We used the same data pre-processing steps as \citet{chen2025} to ensure comparability of results. 5.9\% of individuals had incomplete data and were removed from the dataset. We used 2/3 of the data for fitting the models and 1/3 as a hold out test set. The final dataset included 21 baseline variables recorded at randomization: aspirin allocation (treatment), age, delay between stroke and randomization (in hours), systolic blood pressure, sex, CT performed before randomization, visible infarct on CT, atrial fibrillation, aspirin use within 3 days prior to randomization, and presence or absence of neurological deficits (including face, arm/hand, leg/foot deficits, dysphasia, hemianopia, visuospatial disorder, brainstem or cerebellar signs, and other neurological deficits), as well as consciousness level, stroke subtype, and geographical region. The outcome variable was death or dependence at 6 months.


\medskip

\textbf{Models for ITE estimation: } The aim is to estimate the ITE based on baseline characteristics. As a benchmark, we apply a T-learner logistic regression (following \citet{chen2025}, using the \texttt{stats} package). As a more complex model, we apply a T-learner tuned random forest (using the \texttt{comets} package \citep{comets}), which tunes the number of variables considered for splitting at each node (\texttt{mtry}) and the maximum tree depth (\texttt{max.depth}) using out-of-bag error, with 500 trees. Additionally, we apply an S-learner TRAM-DAG. For the random forest and TRAM-DAG based methods, we additionally scale numerical and dummy encode categorical covariates prior to model training. The transformation function of the outcome is modelled by a complex intercept $h(Y \mid T, \mathbf{X}) = CI(T, \mathbf{X})$, with 4 hidden layers of shape (20, 10, 10, 2). This architecture allows for interaction between the treatment and covariates. Furthermore, batch normalization, ReLU activation, and dropout (0.1) are applied to prevent overfitting and stabilize learning. A validation set comprising 20\% of the training data is used to select the model with the lowest out-of-sample negative log-likelihood, while the test set remains untouched for final evaluation. 

Since the IST is a randomized controlled trial, the full potential of TRAM-DAGs -- designed primarily for use in observational settings -- is not required here, as only the outcome needs to be modeled as a function of baseline patient characteristics. However, applying TRAM-DAGs in this context still allows us to assess its predictive performance and ability to flexibly model interactions between variables.



\medskip

\textbf{Model evaluation: } For validation, since the ground truth is not known, we first rely on calibration plots to assess the general prediction power for the probabilities. Second, we predict the potential outcomes with the trained models to estimate the ITE on the training and test set in terms of the risk difference $\text{ITE}_i = \text{P}(Y_i=1|T=1, \mathbf{X}_i) - \text{P}(Y_i=1|T=0, \mathbf{X}_i)$. For visual validation, we show the densities of the estimated ITEs on both datasets, and the ITE-ATE plots to assess whether the estimated ITEs align with the observed outcomes.










% possible example of unobserved interaction:
% An example could be the psychological condition of a patient which might also affect how the treatment works, this is not a confounder but an effect modifier, and i would assume that this variable is rarely recorede or measured.




\section{Results} \label{sec:results_experiment2}


In this section, we present the results of the ITE estimation on the International Stroke Trial (IST) dataset. The observed average treatment effect (ATE), defined as $\text{P}(Y=1|T=1) - \text{P}(Y=1|T=0)$, was -2.4\% absolute risk reduction on the training set, with a 95\% confidence interval from -4.1\% to -0.6\%. The interval was computed using the Wald method for risk differences, with standard error $\sqrt{p_1 (1 - p_1)/n_1 + p_0 (1 - p_0)/n_0}$
, where $p_1$ and $p_0$ are the event rates in the treated and control groups. On the test set, the observed treatment effect was -0.1\%, with a 95\% confidence interval from -2.6\% to 2.3\%. The ITEs were estimated using three different models: T-learner logistic regression, T-learner tuned random forest, and S-learner TRAM-DAG. The estimated average treatment effect on the test set, calculated as $\text{ATE}_\text{pred}=\text{mean}(\text{ITE}_\text{pred})$, was -2.5\% for the T-learner logistic regression, -2.2\% for the T-learner tuned random forest, and -3.1\% for the S-learner TRAM-DAG. The density of predicted ITEs and the ITE-ATE plots for risk difference per estimated ITE subgroup, including 95\% confidence intervals, are presented in Figures \ref{fig:IST_density_ITE_ATE_glm_tlearner} - \ref{fig:IST_density_ITE_ATE_TRAM_DAG}. Calibration plots are provided in Appendix \ref{sec:calibrations_experiment2}, Figures \ref{fig:calibration_IST_glm} - \ref{fig:calibration_IST_TRAM_DAG}. 




\begin{figure}[htbp]
\centering
\includegraphics[width=0.9\textwidth]{img/results_IST/glm_tlearner_density_ITE_ATE.png}
\caption{Results for the International Stroke Trial (IST) using the T-learner logistic regression. Left: density of predicted ITEs in the training and test sets; Right: observed ATE in terms of risk difference per estimated ITE subgroup.}
\label{fig:IST_density_ITE_ATE_glm_tlearner}
\end{figure}



\begin{figure}[htbp]
\centering
\includegraphics[width=0.9\textwidth]{img/results_IST/IST_tuned_rf_tlearner_density_ITE_ATE.png}
\caption{Results for the International Stroke Trial (IST) using the T-learner tuned random forest. Left: density of predicted ITEs in the training and test sets; Right: observed ATE in terms of risk difference per estimated ITE subgroup.}
\label{fig:IST_density_ITE_ATE_tuned_rf}
\end{figure}


\begin{figure}[htbp]
\centering
\includegraphics[width=0.9\textwidth]{img/results_IST/IST_TRAM_DAG_slearner_density_ITE_ATE.png}
\caption{Results for the International Stroke Trial (IST) using the S-learner TRAM-DAG. Left: density of predicted ITEs in the training and test sets; Right: observed ATE in terms of risk difference per estimated ITE subgroup.}
\label{fig:IST_density_ITE_ATE_TRAM_DAG}
\end{figure}




% enforce that starts after all floats have been displayed
\FloatBarrier

\section{Discussion}

We observed similar results to those reported by \citet{chen2025} when estimating ITEs on the International Stroke Trial dataset across all three models: the T-learner logistic regression, the T-learner tuned random forest, and the S-learner TRAM-DAG. The logistic model showed moderate discrimination in the training set, which did not generalize to the test set, as illustrated by the ITE-ATE plot in Figure~\ref{fig:IST_density_ITE_ATE_glm_tlearner}. The tuned random forest model showed stronger discrimination in the training set but similarly failed to generalize to the test set (Figure~\ref{fig:IST_density_ITE_ATE_tuned_rf}). In contrast, the S-learner TRAM-DAG estimated less heterogeneity than the other two models, as shown in the density plot in Figure~\ref{fig:IST_density_ITE_ATE_TRAM_DAG}, resulting in weak discrimination in both the training and test sets. For all three models, the confidence intervals in the ITE-ATE plots on the test set included the zero line, suggesting no significant effect in any of the estimated ITE subgroups.

\medskip


Poor calibration does not appear to explain the limited ITE performance, as calibration on the test set was good, as shown in Appendix~\ref{sec:calibrations_experiment2}, Figures~\ref{fig:calibration_IST_glm}-\ref{fig:calibration_IST_TRAM_DAG}. However, since the ground truth is unknown, it remains unclear whether the models fail to detect true treatment effect heterogeneity, or whether the heterogeneity is too small, or driven by unobserved effect modifiers. We investigate these possibilities further in Experiment 3, Chapter~\ref{ch:experiment3}.




%%%%%%%%%%%%%%%%%%%%%%%%%%%%%%%%%%%%%%%%%%%%%%%%%%%%%%%%%%%%%%%%%%%%%%
%%%%%%%%%%%%%%%%%%%%%%%%%%%%%%%%%%%%%%%%%%%%%%%%%%%%%%%%%%%%%%%%%%%%%%

% \input{chapter_Exp3}

%%%%%%%%%%%%%%%%%%%%%%%%%%%%%%%%%%%%%%%%%%%%%%%%%%%%%%%%%%%%%%%%%%%%%%
%%%%%%%%%%%%%%%%%%%%%%%%%%%%%%%%%%%%%%%%%%%%%%%%%%%%%%%%%%%%%%%%%%%%%%

% \input{chapter_Exp4}

%%%%%%%%%%%%%%%%%%%%%%%%%%%%%%%%%%%%%%%%%%%%%%%%%%%%%%%%%%%%%%%%%%%%%%
%%%%%%%%%%%%%%%%%%%%%%%%%%%%%%%%%%%%%%%%%%%%%%%%%%%%%%%%%%%%%%%%%%%%%%

% LaTeX file for Chapter 03














\chapter{Results}



% \begin{table}
% 
% \caption{Comparison of confidence intervals for the mean differences across scenarios. The first row of each pair corresponds to the confidence interval from the linear model (lm), while the second row corresponds to the bootstrap method.}
% \centering
% \begin{tabular}[t]{lrr}
% \toprule
%   & Lower Bound & Upper Bound\\
% \midrule
% Scenario 1 (lm) & -0.58231 & -0.54359\\
% Scenario 1 (bootstrap) & -0.58210 & -0.54355\\
% Scenario 2 (lm) & -0.59118 & -0.55371\\
% Scenario 2 (bootstrap) & -0.59091 & -0.55405\\
% Scenario 3 (lm) & -0.06784 & -0.02799\\
% \addlinespace
% Scenario 3 (bootstrap) & -0.06784 & -0.02752\\
% \bottomrule
% \end{tabular}
% \end{table}


\section{Experiment 1: TRAM-DAG (simulation study)}

In this section, we present the results of a simulation study to evaluate the performance of the TRAM-DAG model in a simple scenario as illustrated by the DAG in Figure \ref{fig:dag_and_matrix}. The model was fitted on synthetic data. Figure \ref{fig:exp1_loss_parameters} shows the loss and the estimated parameters for the linear shifts over epochs during training. The loss is minimized during training and the estimated parameters $\beta_{12}$ and $\beta_{13}$ converge to the true values used in the DGP. The linear shift parameters are interpretable part of the model (log odds ratios). 
From the fitted model, we generated samples from the observational distribution, as shown in Figure \ref{fig:exp1_observational_distribution}. Then we drew samples from the interventional distribution, where $X_2 = 1$ is fixed, as shown in Figure \ref{fig:exp1_interventional_distribution}. Fixing $X_2$ leads to a distributional change in $X_3$. The TRAM-DAG model learns the linear shifts ($\beta_{12}$, $\beta_{13}$) and the complex shift ($\text{CS}(X_2)$), which are shown in Figure \ref{fig:exp1_shifts}. Figure \ref{fig:exp1_intercepts} presents the intercepts learned for each of the nodes, with the estimates by the continuous outcomes logistic regression (Colr() function from tram-package \citep{hothorn2018}) function as comparison for the continuous variables and the true values used in the DGP for ordinal variable X3 (3 cut points needed for the 4 levels). Finally, Figure \ref{fig:exp1_counterfactuals} shows the counterfactuals estimated by the TRAM-DAG model for varying values of $X_1$. The counterfactuals are the predicted values of $X_2$ if $X_1$ would have taken other values instead of the observed value. 

\begin{figure}[htbp]
\centering
\includegraphics[width=0.9\textwidth]{img/exp1_loss_parameters.png}
\caption{TRAM-DAG model fitting over 400 epochs for experiment 1. Left: Loss functions on the training set and a separate validation set; Right: Estimated parameters (betas) for the linear shift components over epochs. They converge to the true values.}
\label{fig:exp1_loss_parameters}
\end{figure}



\begin{figure}[htbp]
\centering
\includegraphics[width=0.9\textwidth]{img/exp1_observational_distribution.png}
\caption{Samples by the TRAM-DAG generated from the learned observational against the true observations from the DGP.}
\label{fig:exp1_observational_distribution}
\end{figure}




\begin{figure}[htbp]
\centering
\includegraphics[width=0.9\textwidth]{img/exp1_interventional_distribution.png}
\caption{Samples by the TRAM-DAG generated against the true observations from the interventional distribution, where $X_2 = 1$ is fixed. According to the DAG, this affects a distributional change in $X_3$.}
\label{fig:exp1_interventional_distribution}
\end{figure}



\begin{figure}[htbp]
\centering
\includegraphics[width=0.9\textwidth]{img/exp1_LS_CS.png}
\caption{Linear shift and complex shift learned by the TRAM-DAG. Left: LS($X_1$) on $X_2$; Middle: LS($X_1$) on $X_3$; Right: CS($X_2$) on $X_3$. For visualization, we subtracted $\delta_0 = \text{CS}(0) - f(0)$ from the estimated complex shift CS(X2) to make it comparable to the DGP shift $f(X_2)$}
\label{fig:exp1_shifts}
\end{figure}



\begin{figure}[htbp]
\centering
\includegraphics[width=0.9\textwidth]{img/exp1_baseline_trafo.png}
\caption{Intercepts learned for each of the nodes, with the estimates by the Colr() function for the continuous variables and the true values used in the DGP for ordinal X3. Left: smooth baseline transformation function for continuous X1; Middle: smooth baseline transformation function for continuous X2 ; Right: cut-points as the baseline transformation function for ordinal X3. For the last plot we added $\delta_0 = \text{CS}(0) - f(0)$ to the estimated cut-offs to make them comparable to the true parameters from the DGP.}
\label{fig:exp1_intercepts}
\end{figure}





\begin{figure}[htbp]
\centering
\includegraphics[width=0.9\textwidth]{img/exp1_counterfactuals.png}
\caption{Counterfactuals for $X_2$ estimated with the TRAM-DAG for varying $X_1$. We assumed observations $X_1 = 0.5$, $X_2 = -1.2$, $X_3 = 2$ and determined the counterfactual values for $X_2$ if $X_1$ would have taken other values instead of the observed value.}
\label{fig:exp1_counterfactuals}
\end{figure}


\clearpage


\section{Experiment 2: ITE on International Stroke Trial (IST)} \label{sec:results_experiment2}


In this section, we present the results of the ITE estimation on the International Stroke Trial (IST) dataset. The observed treatment effect $\text{P}(Y=1|T=1) - \text{P}(Y=1|T=0)$ on the training set was -2.4\% absolute risk reduction with a 95\% confidence interval of -4.1\% to -0.6\%. The observed treatment effect on the test set was -0.1\% with a 95\% confidence interval of -2.6\% to 2.3\%. The estimated ITEs were computed using three different models: the T-learner logistic regression, the T-learner tuned random forest, and the S-learner TRAM-DAG. The estimated average treatment effect on the test set as $\text{ATE}_\text{pred}=\text{mean}(\text{ITE}_\text{pred})$ was -2.5\% for the T-learner logistic regression, -2.2\% for the T-learner tuned random forest, and -3.1\% for the S-learner TRAM-DAG. The density of estimated ITEs and ITE-ATE plots in terms of risk difference per estimated ITE subgroup are presented in Figures \ref{fig:IST_density_ITE_ATE_glm_tlearner} - \ref{fig:IST_density_ITE_ATE_TRAM_DAG}. Calibration plots are shown in the Appendix \ref{sec:calibrations_experiment2}, Figures \ref{fig:calibration_IST_glm} - \ref{fig:calibration_IST_TRAM_DAG}. 






\begin{figure}[htbp]
\centering
\includegraphics[width=0.9\textwidth]{img/results_IST/glm_tlearner_density_ITE_ATE.png}
\caption{Results for the International Stroke Trial (IST) with the T-learner logistic regression. Left: density of the predicted ITE in the training and test set; Right: observed ATE in terms of risk difference per estimated ITE subgroup.}
\label{fig:IST_density_ITE_ATE_glm_tlearner}
\end{figure}



\begin{figure}[htbp]
\centering
\includegraphics[width=0.9\textwidth]{img/results_IST/IST_tuned_rf_tlearner_density_ITE_ATE.png}
\caption{Results for the International Stroke Trial (IST) with the T-learner tuned random forest. Left: density of the predicted ITE in the training and test set; Right: observed ATE in terms of risk difference per estimated ITE subgroup.}
\label{fig:IST_density_ITE_ATE_tuned_rf}
\end{figure}


\begin{figure}[htbp]
\centering
\includegraphics[width=0.9\textwidth]{img/results_IST/IST_TRAM_DAG_slearner_density_ITE_ATE.png}
\caption{Results for the International Stroke Trial (IST) with the S-learner TRAM-DAG. Left: density of the predicted ITE in the training and test set; Right: observed ATE in terms of risk difference per estimated ITE subgroup.}
\label{fig:IST_density_ITE_ATE_TRAM_DAG}
\end{figure}



\clearpage

 
\section{Experiment 3: ITE model robustness under RCT conditions (simulation study)} \label{sec:results_experiment3}

In this section, we present the performance of two causal ML models for estimating the ITE under different scenarios. Scenario 1 represents the ideal case where all variables are observed and treatment effects and heterogeneity are large. Scenario 2 uses the same DGP as in scenario 1 but removes the covariate $X_1$, which has a strong interaction effect with the treatment, from the dataset and treat it as unobserved. Finally, for scenario 3 the coefficients for the direct and interaction treatment effects are weakened, so that heterogeneity is low. All variables are observed again in the last scenario. In each scenario, we applied the T-learner logistic regression and the T-learner tuned random forest. The results of the models on the three scenarios are presented in Figures \ref{fig:fully_observed_glm_tlearner} to \ref{fig:small_interaction_tuned_rf_tlearner}.



\subsection{Scenario (1): Fully observed, large effects}



\begin{figure}[htbp]
\centering
\includegraphics[width=0.35\textwidth]{img/results_ITE_simulation/simulation_observed.png}
\caption{DAG for scenario (1), where all variables are observed and there are strong treatment and interaction effects. The numbers indicate the coefficients on the log-odds-scale. Red: interaction effects between treatment ($T$) and covariates ($X_1$ and $X_2$) on the outcome ($Y$).}
\label{fig:fully_observed_dag}
\end{figure}


\begin{figure}[htbp]
\centering
\includegraphics[width=0.9\textwidth]{img/results_ITE_simulation/fully_observed_glm_tlearner.png}
\caption{Results with the T-learner logistic regression in scenario (1) when the DAG is fully observed and there are strong treatment and interaction effects. Left: true vs. predicted probabilities for $\text{P}(Y=1 \mid X, T)$; Middle: true vs. predicted ITEs; Right: observed ATE in terms of risk difference per estimated ITE subgroup.}
\label{fig:fully_observed_glm_tlearner}
\end{figure}


\begin{figure}[htbp]
\centering
\includegraphics[width=0.9\textwidth]{img/results_ITE_simulation/fully_observed_tuned_rf_tlearner.png}
\caption{Results with the T-learner tuned random forest in scenario (1) when the DAG is fully observed, strong effects. Left: true vs. predicted probabilities for $\text{P}(Y=1 \mid X, T)$; Middle: true vs. predicted ITEs; Right: observed ATE in terms of risk difference per estimated ITE subgroup.}
\label{fig:fully_tuned_rf_tlearner}
\end{figure}


\clearpage



\subsection{Scenario (2): unobserved interaction}

\begin{figure}[htbp]
\centering
\includegraphics[width=0.35\textwidth]{img/results_ITE_simulation/simulation_unobserved.png}
\caption{DAG for scenario (2), where there are strong treatment and interaction effects, but variable $X1$ is not observed. The numbers indicate the coefficients on the log-odds-scale. Red: interaction effects between treatment ($T$) and covariates ($X_1$ and $X_2$) on the outcome ($Y$).}
\label{fig:unobserved_interaction_dag}
\end{figure}



\begin{figure}[htbp]
\centering
\includegraphics[width=0.9\textwidth]{img/results_ITE_simulation/unobserved_interaction_glm_tlearner.png}
\caption{Results with the T-learner logistic regression in scenario (2) when there are strong treatment and interaction effects, but variable $X_1$ is not observed. Left: true vs. predicted probabilities for $\text{P}(Y=1 \mid X, T)$; Middle: true vs. predicted ITEs; Right: observed ATE in terms of risk difference per estimated ITE subgroup.}
\label{fig:unobserved_interaction_glm_tlearner}
\end{figure}



\begin{figure}[htbp]
\centering
\includegraphics[width=0.9\textwidth]{img/results_ITE_simulation/unobserved_interaction_tuned_rf_tlearner.png}
\caption{Results with the T-learner tuned random forest in scenario (2) when there are strong treatment and interaction effects, but variable $X_1$ is not observed. Left: true vs. predicted probabilities for $\text{P}(Y=1 \mid X, T)$; Middle: true vs. predicted ITEs; Right: observed ATE in terms of risk difference per estimated ITE subgroup.}
\label{fig:unobserved_interaction_tuned_rf_tlearner}
\end{figure}


\clearpage

\subsection{Scenario (3): Fully observed, small effects}

\begin{figure}[htbp]
\centering
\includegraphics[width=0.35\textwidth]{img/results_ITE_simulation/simulation_small_effects.png}
\caption{DAG for scenario (3), where all variables are observed and there are weak treatment and interaction effects. The numbers indicate the coefficients on the log-odds-scale. Red: interaction effects between treatment ($T$) and covariates ($X_1$ and $X_2$) on the outcome ($Y$).}
\label{fig:small_interaction_dag}
\end{figure}




\begin{figure}[htbp]
\centering
\includegraphics[width=0.9\textwidth]{img/results_ITE_simulation/small_interaction_glm_tlearner.png}
\caption{Results with the T-learner logistic regression in scenario (3) when the DAG is fully observed and there are weak treatment and interaction effects. Left: true vs. predicted probabilities for $\text{P}(Y=1 \mid X, T)$; Middle: true vs. predicted ITEs; Right: observed ATE in terms of risk difference per estimated ITE subgroup.}
\label{fig:small_interaction_glm_tlearner}
\end{figure}




\begin{figure}[htbp]
\centering
\includegraphics[width=0.9\textwidth]{img/results_ITE_simulation/small_interaction_tuned_rf_tlearner.png}
\caption{Results with the T-learner tuned random forest in scenario (3) when the DAG is fully observed and there are weak treatment and interaction effects. Left: true vs. predicted probabilities for $\text{P}(Y=1 \mid X, T)$; Middle: true vs. predicted ITEs; Right: observed ATE in terms of risk difference per estimated ITE subgroup.}
\label{fig:small_interaction_tuned_rf_tlearner}
\end{figure}


\clearpage


\section{Experiment 4: ITE estimation with TRAM-DAGs (simulation study)}

First, we present the results for scenario (1) with a direct and interaction effect. Then, we present the results for scenario (2) with a direct effect but no interaction effects, and finally, scenario (3) with interaction effects but no direct effect of the treatment. For each scenario, we compare the results in an observational setting with confounded treatment allocation and in a randomized controlled trial (RCT) setting without confounders. We also compare the average treatment effect (ATE), which can directly be calculated in the RCT, with the ATE based on the estimated individualized treatment effects. If the estimated ITEs are unbiased, they should be a good estimate of the ATE. All ITEs presented in this section are technically quantile treatment effects (QTEs) based on the 0.5-quantile of the potential outcomes. For simplicity we will refer to them as ITEs in the following.


\subsection{Scenario (1): Direct and interaction effects} 

Scenario (1) included a direct effect of the treatment on the outcome and an additional interaction effect of the treatment with the covariates X2 and X3. A train and test set were generated with 20'000 observations each. In the observational setting, the treatment allocation was confounded by the covariates X1 and X2.  In the train set, $38.6$\% of patients were in the control group and $61.4$\% were in the treatment group. This ratio was similar in the test set. In the RCT setting treatment allocation was randomized. In the train set $49.8$\% individuals were in the control group and $50.2$\% in the treatment group. In the test set $50.2$\% were in the control group and $49.8$\% in the treatment group. Figure \ref{fig:scenario1_ite_distribution_dgp} illustrates the true ITE distribution that resulted from the DGP. Due to the interaction effects, there is some heterogeneity in the ITE distribution. Figure \ref{fig:scenario1_sampling_distributions_vertical} shows the marginal distributions of all variables according to the DGP and the estimates of the fitted TRAM-DAG. Figure \ref{fig:scenario1_outcome_distributions} shows the distribution of the outcome under the do(Tr=0) and do(Tr=1) interventions. The fitted model was applied to estimate the ITEs in terms of the difference in medians of the potential outcomes. The resulting density of the estimated ITEs compared to the true ITEs according to the DGP is shown in Figure \ref{fig:scenario1_ite_densities_train_test}. Across both settings, the densities of the estimated ITEs are close to the true densities in both the training and test datasets. Figure \ref{fig:scenario1_ite_scatter_train_test} shows the scatterplots of true against estimated ITEs. Finally, Figure \ref{fig:scenario1_ite_cATE} displays the ITE-ATE plot where the ATE is computed as the difference in medians of the observed outcome under the treatments within the respective ITE-subgroups The trends observed in the training and test sets are consistent.

The average treatment effect (ATE) is presented in Table \ref{tab:scenario1_ate_comparison}. In the RCT setting in the training set, the difference in means of the outcomes in the two treatment groups was $-0.563$ with a confidence interval of $-0.582$ to $-0.543$. The ATE in terms of the difference in medians of the observed outcomes was $-0.626$. Also in the training set, the ATE in terms of the mean of the true ITEs was $-0.62$ and the ATE in terms of the mean of the estimated ITEs was $-0.619$. All measures, including the ones from the test datasets, are shown in Table \ref{tab:scenario1_ate_comparison}.

NOTE: also add CIs in the table with the ATEs?

\begin{table}[htbp]
\centering
\small
\caption{Scenario (1), including direct and interaction effects: Comparison of ATE measures across train and test sets for the observational and RCT setting.}
\label{tab:scenario1_ate_comparison}
\begin{tabular}{l c c c c}
\toprule
\textbf{Measure} & \multicolumn{2}{c}{\textbf{Observational}} & \multicolumn{2}{c}{\textbf{RCT}} \\
\cmidrule(lr){2-3} \cmidrule(lr){4-5}
 & \textbf{Train} & \textbf{Test} & \textbf{Train} & \textbf{Test} \\
\midrule
ATE as $\text{mean}(\text{Y}_\text{observed}^{(1)}) - \text{mean}(\text{Y}_\text{observed}^{(0)})$ & NA & NA & -0.563 & -0.563 \\
ATE as $\text{median}(\text{Y}_\text{observed}^{(1)}) - \text{median}(\text{Y}_\text{observed}^{(0)})$  & NA & NA & -0.626 & -0.638 \\
ATE as mean(ITE$_\text{true}$)  & -0.62 & -0.622 & -0.62 & -0.622 \\
ATE as mean(ITE$_\text{estimated}$) & -0.617 & -0.62 & -0.619 & -0.622 \\
\bottomrule
\end{tabular}
\end{table}




\begin{figure}[htbp]
\centering
\includegraphics[width=0.45\textwidth]{img/results/observ_scenario1_ite_distribution_dgp.png}
\includegraphics[width=0.45\textwidth]{img/results/rct_scenario1_ite_distribution_dgp.png}
\caption{True ITE distribution resulting from the DGP for scenario (1) with direct and interaction effects. The true ITEs are identical in the observational and in the RCT setting, since they depend on the potential outcomes under both treatment allocations. Left: Observational; Right: RCT setting.}
\label{fig:scenario1_ite_distribution_dgp}
\end{figure}



\begin{figure}[htbp]
\centering
\includegraphics[width=0.45\textwidth]{img/results/observ_scenario1_sampling_distributions_vertical.png}
\includegraphics[width=0.45\textwidth]{img/results/rct_scenario1_sampling_distributions_vertical.png}
\caption{Marginal distributions of DGP variables and fitted TRAM-DAG samples for scenario (1) with direct and interaction effects. The distributions shown as observed (Obs), under control intervention (Do $X4=0$) and under treatment intervention (Do $X4=1$). Left: Observational; Right: RCT setting.}
\label{fig:scenario1_sampling_distributions_vertical}
\end{figure}

\begin{figure}[htbp]
\centering
\includegraphics[width=0.45\textwidth]{img/results/observ_scenario1_X7_treatment_densities.png}
\includegraphics[width=0.45\textwidth]{img/results/rct_scenario1_X7_treatment_densities.png}
\caption{Distributions of the outcome variable (X7) under treatment and control interventions for scenario (1), including direct and interaction effects. This plot is a higher resolution view of the X7 panels (Do $X4=0$) and (Do $X4=1$) from Figure \ref{fig:scenario1_sampling_distributions_vertical}. Left: Observational; Right: RCT setting.}
\label{fig:scenario1_outcome_distributions}
\end{figure}




\begin{figure}[htbp]
\centering
\includegraphics[width=0.45\textwidth]{img/results/observ_scenario1_ITE_densities_train_test.png}
\includegraphics[width=0.45\textwidth]{img/results/rct_scenario1_ITE_densities_train_test.png}
\caption{Densities of estimated ITEs compared to the true ITEs in the training and test datasets for scenario (1), including direct and interaction effects. Left: Observational; right: RCT setting.}
\label{fig:scenario1_ite_densities_train_test}
\end{figure}






\begin{figure}[htbp]
\centering
\includegraphics[width=0.45\textwidth]{img/results/observ_scenario1_ITE_scatter_train_test.png}
\includegraphics[width=0.45\textwidth]{img/results/rct_scenario1_ITE_scatter_train_test.png}
\caption{Scatterplots of estimated ITEs compared to the true ITEs in the training and test datasets for scenario (1), including direct and interaction effects. Left: Observational; right: RCT setting.}
\label{fig:scenario1_ite_scatter_train_test}
\end{figure}




\begin{figure}[htbp]
\centering
\includegraphics[width=0.45\textwidth]{img/results/observ_scenario1_ITE_cATE.png}
\includegraphics[width=0.45\textwidth]{img/results/rct_scenario1_ITE_cATE.png}
\caption{ITE-ATE plot for scenario (1), including direct and interaction effects. Individuals are grouped into bins according to the estimated ITE and in each bin the ATE is calculated as the difference in medians of the observed outcomes under the treatments. 95\% bootstrap confidence intervals indicate the uncertainty. Left: Observational; right: RCT setting.}
\label{fig:scenario1_ite_cATE}
\end{figure}



% start a new page
\clearpage


\subsection{Scenario (2): With direct but no interaction effects}

Scenario (2) included a direct effect of the treatment on the outcome and coefficients of the interaction effects are set to zero. This results in less heterogeneity of ITE compared to scenario (1) as shown in Figure \ref{fig:scenario2_ite_distribution_dgp}. The observational and interventional densities sampled by the fitted TRAM-DAG are aligned with the true densities according to the DGP as illustrated in Figures \ref{fig:scenario2_sampling_distributions_vertical} and \ref{fig:scenario2_outcome_distributions}. A notable discrepancy in variance exists between the estimated and true ITEs, as illustrated in Figures \ref{fig:scenario2_ite_densities_train_test} and \ref{fig:scenario2_ite_scatter_train_test}. The ITE-ATE plot in Figure \ref{fig:scenario2_ite_cATE} shows a less informative view compared to scenario (1). Table \ref{tab:scenario2_ate_comparison} presents the ATE measures for scenario (2). In the test set of the RCT setting, the ATE in terms of the difference in medians of the observed outcomes was $-0.639$. In contrast, the ATE based on the estimated ITEs in the same dataset was $-0.586$.


\begin{table}[htbp]
\centering
\small
\caption{Scenario (2), including a direct treatment but no interaction effects: Comparison of ATE measures across train and test sets for the observational and RCT setting.}
\label{tab:scenario2_ate_comparison}
\begin{tabular}{l c c c c}
\toprule
\textbf{Measure} & \multicolumn{2}{c}{\textbf{Observational}} & \multicolumn{2}{c}{\textbf{RCT}} \\
\cmidrule(lr){2-3} \cmidrule(lr){4-5}
 & \textbf{Train} & \textbf{Test} & \textbf{Train} & \textbf{Test} \\
\midrule
ATE as $\text{mean}(\text{Y}_\text{observed}^{(1)}) - \text{mean}(\text{Y}_\text{observed}^{(0)})$ & NA & NA & -0.569 & -0.572 \\
ATE as $\text{median}(\text{Y}_\text{observed}^{(1)}) - \text{median}(\text{Y}_\text{observed}^{(0)})$  & NA & NA & -0.629 & -0.639 \\
ATE as mean(ITE$_\text{true}$)  & -0.633 & -0.633 & -0.633 & -0.633 \\
ATE as mean(ITE$_\text{estimated}$) & -0.645 & -0.644 & -0.587 & -0.586 \\
\bottomrule
\end{tabular}
\end{table}



\begin{figure}[htbp]
\centering
\includegraphics[width=0.45\textwidth]{img/results/observ_scenario2_ite_distribution_dgp.png}
\includegraphics[width=0.45\textwidth]{img/results/rct_scenario2_ite_distribution_dgp.png}
\caption{True ITE distribution resulting from the DGP for scenario (2), including a direct treatment but no interaction effects. The true ITEs are identical in the observational and in the RCT setting, since they depend on the potential outcomes under both treatment allocations. Left: Observational; Right: RCT setting.}
\label{fig:scenario2_ite_distribution_dgp}
\end{figure}



\begin{figure}[htbp]
\centering
\includegraphics[width=0.45\textwidth]{img/results/observ_scenario2_sampling_distributions_vertical.png}
\includegraphics[width=0.45\textwidth]{img/results/rct_scenario2_sampling_distributions_vertical.png}
\caption{Marginal distributions of DGP variables and fitted TRAM-DAG samples for scenario (2), including a direct treatment but no interaction effects. The distributions shown as observed (Obs), under control intervention (Do $X4=0$) and under treatment intervention (Do $X4=1$). Left: Observational; Right: RCT setting.}
\label{fig:scenario2_sampling_distributions_vertical}
\end{figure}

\begin{figure}[htbp]
\centering
\includegraphics[width=0.45\textwidth]{img/results/observ_scenario2_X7_treatment_densities.png}
\includegraphics[width=0.45\textwidth]{img/results/rct_scenario2_X7_treatment_densities.png}
\caption{Distributions of the outcome variable (X7) under treatment and control interventions for scenario (2), including a direct treatment but no interaction effects. This plot is a higher resolution view of the X7 panels (Do $X4=0$) and (Do $X4=1$) from Figure \ref{fig:scenario2_sampling_distributions_vertical}. Left: Observational; Right: RCT setting.}
\label{fig:scenario2_outcome_distributions}
\end{figure}




\begin{figure}[htbp]
\centering
\includegraphics[width=0.45\textwidth]{img/results/observ_scenario2_ITE_densities_train_test.png}
\includegraphics[width=0.45\textwidth]{img/results/rct_scenario2_ITE_densities_train_test.png}
\caption{Densities of estimated ITEs compared to the true ITEs in the training and test datasets for scenario (2), including a direct treatment but no interaction effects. Left: Observational; right: RCT setting.}
\label{fig:scenario2_ite_densities_train_test}
\end{figure}






\begin{figure}[htbp]
\centering
\includegraphics[width=0.45\textwidth]{img/results/observ_scenario2_ITE_scatter_train_test.png}
\includegraphics[width=0.45\textwidth]{img/results/rct_scenario2_ITE_scatter_train_test.png}
\caption{Scatterplots of estimated ITEs compared to the true ITEs in the training and test datasets for scenario (2), including a direct treatment but no interaction effects. Left: Observational; right: RCT setting.}
\label{fig:scenario2_ite_scatter_train_test}
\end{figure}




\begin{figure}[htbp]
\centering
\includegraphics[width=0.45\textwidth]{img/results/observ_scenario2_ITE_cATE.png}
\includegraphics[width=0.45\textwidth]{img/results/rct_scenario2_ITE_cATE.png}
\caption{ITE-ATE plot for scenario (2), including a direct treatment but no interaction effects. Individuals are grouped into bins according to the estimated ITE and in each bin the ATE is calculated as the difference in medians of the observed outcomes under the treatments. 95\% bootstrap confidence intervals indicate the uncertainty. Left: Observational; right: RCT setting.}
\label{fig:scenario2_ite_cATE}
\end{figure}



\clearpage 
¨
\subsection{Scenario (3): No direct but with interaction effects}

Scenario (3) included no direct effect of the treatment on the outcome but it included interaction effects of the treatment with the covariates X2 and X3. Compared to scenario (1), when excluding the direct effect of the treatment, the distribution of ITEs is more centered as shown in Figure \ref{fig:scenario3_ite_distribution_dgp}. The ATE in terms of the mean difference in the test set of the RCT setting is $-0.048$ with a confidence interval of $-0.068$ to $-0.028$. 



\begin{table}[htbp]
\centering
\small
\caption{Scenario (3), without direct treatment effect but including interaction effects: Comparison of ATE measures across train and test sets for the observational and RCT setting.}
\label{tab:scenario3_ate_comparison}
\begin{tabular}{l c c c c}
\toprule
\textbf{Measure} & \multicolumn{2}{c}{\textbf{Observational}} & \multicolumn{2}{c}{\textbf{RCT}} \\
\cmidrule(lr){2-3} \cmidrule(lr){4-5}
 & \textbf{Train} & \textbf{Test} & \textbf{Train} & \textbf{Test} \\
\midrule
ATE as $\text{mean}(\text{Y}_\text{observed}^{(1)}) - \text{mean}(\text{Y}_\text{observed}^{(0)})$ & NA & NA & -0.048 & -0.048 \\
ATE as $\text{median}(\text{Y}_\text{observed}^{(1)}) - \text{median}(\text{Y}_\text{observed}^{(0)})$ & NA & NA & -0.048 & -0.059 \\
ATE as mean(ITE$_\text{true}$)  & -0.065 & -0.068 & -0.065 & -0.068 \\
ATE as mean(ITE$_\text{estimated}$) & -0.059 & -0.061 & -0.051 & -0.053 \\
\bottomrule
\end{tabular}
\end{table}



\begin{figure}[htbp]
\centering
\includegraphics[width=0.45\textwidth]{img/results/observ_scenario3_ite_distribution_dgp.png}
\includegraphics[width=0.45\textwidth]{img/results/rct_scenario3_ite_distribution_dgp.png}
\caption{True ITE distribution resulting from the DGP for scenario (3), without direct treatment effect but including interaction effects. The true ITEs are identical in the observational and in the RCT setting, since they depend on the potential outcomes under both treatment allocations. Left: Observational; Right: RCT setting.}
\label{fig:scenario3_ite_distribution_dgp}
\end{figure}



\begin{figure}[htbp]
\centering
\includegraphics[width=0.45\textwidth]{img/results/observ_scenario3_sampling_distributions_vertical.png}
\includegraphics[width=0.45\textwidth]{img/results/rct_scenario3_sampling_distributions_vertical.png}
\caption{Marginal distributions of DGP variables and fitted TRAM-DAG samples for scenario (3), without direct treatment effect but including interaction effects. The distributions shown as observed (Obs), under control intervention (Do $X4=0$) and under treatment intervention (Do $X4=1$). Left: Observational; Right: RCT setting.}
\label{fig:scenario3_sampling_distributions_vertical}
\end{figure}

\begin{figure}[htbp]
\centering
\includegraphics[width=0.45\textwidth]{img/results/observ_scenario3_X7_treatment_densities.png}
\includegraphics[width=0.45\textwidth]{img/results/rct_scenario3_X7_treatment_densities.png}
\caption{Distributions of the outcome variable (X7) under treatment and control interventions for scenario (3), without direct treatment effect but including interaction effects. This plot is a higher resolution view of the X7 panels (Do $X4=0$) and (Do $X4=1$) from Figure \ref{fig:scenario3_sampling_distributions_vertical}. Left: Observational; Right: RCT setting.}
\label{fig:scenario3_outcome_distributions}
\end{figure}




\begin{figure}[htbp]
\centering
\includegraphics[width=0.45\textwidth]{img/results/observ_scenario3_ITE_densities_train_test.png}
\includegraphics[width=0.45\textwidth]{img/results/rct_scenario3_ITE_densities_train_test.png}
\caption{Densities of estimated ITEs compared to the true ITEs in the training and test datasets for scenario (3), without direct treatment effect but including interaction effects. Left: Observational; right: RCT setting.}
\label{fig:scenario3_ite_densities_train_test}
\end{figure}






\begin{figure}[htbp]
\centering
\includegraphics[width=0.45\textwidth]{img/results/observ_scenario3_ITE_scatter_train_test.png}
\includegraphics[width=0.45\textwidth]{img/results/rct_scenario3_ITE_scatter_train_test.png}
\caption{Scatterplots of estimated ITEs compared to the true ITEs in the training and test datasets for scenario (3), without direct treatment effect but including interaction effects. Left: Observational; right: RCT setting.}
\label{fig:scenario3_ite_scatter_train_test}
\end{figure}




\begin{figure}[htbp]
\centering
\includegraphics[width=0.45\textwidth]{img/results/observ_scenario3_ITE_cATE.png}
\includegraphics[width=0.45\textwidth]{img/results/rct_scenario3_ITE_cATE.png}
\caption{ITE-ATE plot for scenario (3), without direct treatment effect but including interaction effects. Individuals are grouped into bins according to the estimated ITE and in each bin the ATE is calculated as the difference in medians of the observed outcomes under the treatments. 95\% bootstrap confidence intervals indicate the uncertainty. Left: Observational; right: RCT setting.}
\label{fig:scenario3_ite_cATE}
\end{figure}


  % not used anymore

%%%%%%%%%%%%%%%%%%%%%%%%%%%%%%%%%%%%%%%%%%%%%%%%%%%%%%%%%%%%%%%%%%%%%%
%%%%%%%%%%%%%%%%%%%%%%%%%%%%%%%%%%%%%%%%%%%%%%%%%%%%%%%%%%%%%%%%%%%%%%


% LaTeX file for Chapter 04



% \chapter{Discussion and Outlook}

\chapter{Discussion}


here general discussion, eg. importance of calibration




%%%%%%%%%%%%%%%%%%%%%%%%%%%%%%%%%%%%%%%%%%%%%%%%%%%%%%%%%%%%%%%%%%%%%%
%%%%%%%%%%%%%%%%%%%%%%%%%%%%%%%%%%%%%%%%%%%%%%%%%%%%%%%%%%%%%%%%%%%%%%

% % LaTeX file for Chapter 05


\chapter{Conclusions}



In this thesis, we further investigated the application of TRAM-DAGs as a flexible approach to estimate structural equations in a known DAG. We explained how to incorporate ordinal predictors, how to model interactions, and what the scaling of variables implies for interpretability. Furthermore, we explored the estimation of individualized treatment effects (ITEs), showing that TRAM-DAGs can also be applied to estimate ITEs in relatively complex DAG structures. In simulation experiments, we examined potential limitations and challenges in ITE estimation. 

\medskip

Our findings included that TRAM-DAGs were able to successfully recover structural equations when the DAG was fully known and all variables were observed. They also worked well for ITE estimation in simulation settings. The simulation experiments further revealed limitations in ITE estimation, especially in the presence of unobserved effect modifiers. We concluded that unmeasured effect-modifying variables pose a significant challenge and that the ignorability assumption alone may not be enough to ensure unbiased estimates. This or weak treatment effect heterogeneity might explain why ITE estimation failed in the real-world application on the International Stroke Trial. We also found that proper calibration of causal machine learning models is important to achieve accurate ITE estimates but that calibration alone may not be sufficient for valid predictions.


\medskip

TRAM-DAGs offer several advantages. The model inherently knows when to adjust for covariates based on the DAG structure and the learned functions. Structural causal models, in contrast to classical regression approaches, account for all known relationships and can consistently address confounding. Classical regression models risk adjusting for the wrong covariates, which may lead to biased estimates. TRAM-DAGs are generative causal models that, once fitted to a correct DAG, allow sampling from observational, interventional, and counterfactual distributions. Their ability to combine flexible components with interpretable structure makes them well suited for practical use cases where both predictive power and transparency matter.

\medskip

However, there are also some limitations. While we aimed to make the simulation scenarios as realistic as possible while still retaining some interpretability, they may not fully reflect the complexity of real-world data. However, applying and evaluating models like TRAM-DAGs on real data for causal questions such as ITE estimation is inherently difficult, as the true effects are usually unknown. TRAM-DAGs also rely on neural networks, which require time to train, depending on network complexity, sample size, and computational resources. And although TRAM-DAGs offer flexibility, we still need to make assumptions -- for example, about the scale on which conditional effects occur -- if we want to retain some level of interpretability.

\medskip

Future work could apply TRAM-DAGs to other real-world datasets, potentially also including semi-structured data, to fully exploit the potential of their modular neural network structure. It would also be valuable to further investigate ITE estimation in the presence of unmeasured interaction variables.

\medskip
Overall, this thesis contributes to the growing field of causal inference, especially in observational data and personalized interventions. We hope to have provided some insights into the capabilities of neural causal models and the challenges of ITE estimation.

% 
% - summary of thesis
% 
% In this thesis, we further investigated the application of TRAM-DAGs as a flexible approach to estimate structural equations in a given DAG. We also showed explained how to incorporate ordinal predictors, how to model interacitons, or what scaling of variables implies on interpretability. Furhtermore, we dived into the estimation of indiviudalized treatment effects, where we showed that TRAM-DAGs can be used as estimation tool also in relatively complex DAG structures.  In simulation experiments we further analyzed possible limitations or difficulties in ITE estimation - where we came to the conclusion that unmeasured effect modifying variables pose a significant challenge and that the ignorability assumption may not be enough to ensure unbiased estimates.
% 
% 
% - key findings
% 
% TRAM-DAGs could successfully recover structural equations if the dag is fully known and fully observed, and that it can also very well be applied for ITE estimation.
% In simulation experiments we further analyzed possible limitations or difficulties (pitfalls) in ITE estimation - where we came to the conclusion that unmeasured effect modifying variables pose a significant challenge.
% this, or weak heterogeneity, may have been a reason why ITE estimation failed in the real world application of the International Stroke Trial. also Calibration of causal ML models is key to achieve an accurate ITE estimation, however, not enough to ensure valid predictions.
% 
% - implications
% 
% TRAM-DAGs  are good because:
% 
% My comments:
% Benefit of TRAM-DAGS: our model knows when to adjust due to the causal graph and learned functions. With classical regression approaches, there is the danger of wrongly adjusting for covariates and therefore obtaining misleading parameter estimates for inference. A tram dag is a generative causal model because once fitted it can be used to generate samples from the distributions (including interventional and counterfactual.). Hybrid modelling (interpretable, flexible, etc) is a unique capability of this model, to our knowledge. " chatgpt: combining the robustness of structural causal modeling with the representational power of modern neural networks. Furthermore, the ability to selectively impose interpretability on certain components makes the model suitable for real-world tasks requiring both transparency and flexibility."
% 
% Also include something like the following from \citep{nichols2007} as another reason to apply SCM's (which TRAM-DAGs are) instead of just adjusting. like that we just fit the whole thing what is known, and it basically takes care of counfounders and everything.
%  "The literature on structural equations models is extensive, and a system of equations may encode a complicated conceptual causal model,
% with many causal arrows drawn to and from many variables. The present exercise of
% identifying the causal impact of some limited set of variables XT on a single outcome
% y can be seen as restricting our attention in such a complicated system to just one
% equation, and identifying just some subset of causal effects."
% 
% and they are highly customizable depending on the usecase. they can be used for ITE estimation.
% 
% - limitations
% 
% Although we tried to make simulation scenarios as realistic as possible, while maintaining interpretability, they may not fully capture the complexities of real-world data. And analyzing models on real-world data for causal questions such as ITE estimation is inherently challenging as the ground truth is unknown.
% 
% TRAM-DAGs may offer flexibility and interpretability, but relying on neural networks, the learning process takes time, depending e.g. on network complexity, number of samples or computational power available for training.
% 
% Although very flexible we still have to make assumptions if we maintain some interpretability -- such as the scale on which the conditional effects of predictors may occur.
% 
% 
% - future directions
% 
% Future work could explore the application of TRAM-DAGs to real-world datasets, possibly also with semi-structured data to fully exploit TRAM-DAGs potential  as a modular neutal network structure. It may also be valuable to further study ITE estimation in presence of unmeasured interaction variables.
% 
% 
% - closing words
% 
% Overall, this thesis contributes to the evolving field of causal inference under observational data and in context of personalized interventions. We hope to having provided some insights on the capabilities of neural causal models in this regards and on possible Issues in ITE estimation.
% 
% 




% Discuss all of the following subjects also include literature and reasoning and explanation (maybe also move a part to methods) (include basically anything that we encountered): ghost interactions (literature from hoogland paper (2021?) and non-collapsibility from susanne and torsten). Overfitting of certain models in differenct settings. We need enough true heterogeneity so that the model can actually detect something, complex models might just overfit else. And the relevant variables observed, else models seem to have problems to allocate the observed heterogeneity. complex models tend to predict too much heterogeneity (if there is no heterogeneity), however in the case where an interaction variable was not observed, also the complex model predicted too narrow heterogeneity (see tuned-rf scenario 2 unobserved). Talk about S learner and T learner and the difference in performance or things we have to consider? Tram dag s learner seems to also work well when fully observed. We used QTE for last experiment, but Expected potential outcomes would also be possible with sampling or numerical integration (lucas mentioned that.) ITE based on Expected is certainly more generally known etc. but in certain applictions /problems QTE might be a better choice. Because of computational simplicity we used QTE.



% Benefit of TRAM-DAGS: our model knows when to adjust due to the causal graph and learned functions. With classical regression approaches, there is the danger of wrongly adjusting for covariates and therefore obtaining misleading parameter estimates for inference. A tram dag is a generative causal model because once fitted it can be used to generate samples from the distributions (including interventional and counterfactual.). Hybrid modelling (interpretable, flexible, etc) is a unique capability of this model, to our knowledge. " chatgpt: combining the robustness of structural causal modeling with the representational power of modern neural networks. Furthermore, the ability to selectively impose interpretability on certain components makes the model suitable for real-world tasks requiring both transparency and flexibility."

%  for interpretation see pearl book 2009 p. 366. the key is to say leaving all other variables "untouched" and not "constant". he also talks about the connection to the do-operator.

% ignorability alone is not sufficient for the estimation of individual effects;
% 
% % - Maybe a good conclusion: because this problematic with missing effect modifiers in RCT data can be a motivation to work with observational data where the dag is very detailed specified with all confounders and interactions, then a tram-dag can be applied. However, there we also have the problem, that important variables are probably also not known/measured...
% 
% 
% %  i think nichols2007 (or maybe the one russian with IV and QTE). stated that while structural causal models provide  the full picture, approaches that aim to estimate treatment effects by sole conditioning/controling for certain variables only look at a very isolated part in the system. this could be an argument for TRAM-DAGs
% 
% 
% % we answered the four research questions posed in the introduction. We showed tram dags capabilities and how it works with ordinal predictors (appendix), how it models interactions (appendix, experiment 2 and 4).. 
% refer to all 4 reseatch questions briefely to show accomplishements.


% The poor performance in the IST dataset was likely due to true weak heterogeneity or due to unobserved variables. We come to this conclusion because of our simulations in experiment 3 that revealed these possible problems.


% We showed how TRAM-DAGS can be applied do estimate the causal relationships in a given fully observed DAG. We pointed out the importance of individualized treatment effects, for example in personnalized medicine or targeted marketing. Calibration of causal ML models is key to achieve an accurate ITE estimation. Also the trade off between complexity and generalizability becomes more important in this application compared to sole predictive modelling. We pointed out potential pitfalls that can emerge in real world settings and should be paid attention towards. These can be for example too little heterogeneity or general poor effect of the treatment, or the fact that there could be unobserved effect modifiers (treatment-covariate interactions). In terms of effect modifiers, methods in literature have already been proposed such as instrumental variables (IV) or Negative Controls (?) where additional variables in a special dependency to the treatment and exposure are used to adjust for unobserved variables (confounders or effect modifiers?). However, it strongly depends on the setting and it is not guaranteed that there exist such supporting variables. We claim that if we know the structure of the DAG, with TRAM-DAGs we can estimate the ITE regardless if we have a RCT or observational data. The only requirement is that the DAG is correct and fully observed, i.e. no unobserved confounders or effect modifiers exist. And since the average treatment effect (ATE) is the average of the individual treatment effects (ITE), we can also estimate the ATE from the ITEs. This implies that running an expensive RCT is not necessary if we have a good observational dataset and know the DAG structure. Our last experiment supports this claim. We used the medians of the potential outcomes to calculate the ITE, however, if the ITE was calculated based on the expected values, it would be directly comparable to the ATE from the RCT in terms of the difference in means, which might be a more classical measure. 
 % not used anymore

%%%%%%%%%%%%%%%%%%%%%%%%%%%%%%%%%%%%%%%%%%%%%%%%%%%%%%%%%%%%%%%%%%%%%%
%%%%%%%%%%%%%%%%%%%%%%%%%%%%%%%%%%%%%%%%%%%%%%%%%%%%%%%%%%%%%%%%%%%%%%

\cleardoublepage
\phantomsection
\addtocontents{toc}{\protect \vspace*{10mm}}
\addcontentsline{toc}{chapter}{\bfseries Bibliography}


\bibliographystyle{mywiley} 
\bibliography{biblio}

%%%%%%%%%%%%%%%%%%%%%%%%%%%%%%%%%%%%%%%%%%%%%%%%%%%%%%%%%%%%%%%%%%%%%%
%%%%%%%%%%%%%%%%%%%%%%%%%%%%%%%%%%%%%%%%%%%%%%%%%%%%%%%%%%%%%%%%%%%%%%

% appendix

% LaTeX file for Chapter 06




\chapter{Appendix}


To do:

- include results of ITE estimation with not tuned RF for scenario 1 (fully observed,strong effects) to show the importance that models should be well calibrated (in this case tuned) to yield good results

- inlcude Example of ITE simulation with tram dags with complex shift: $theta + LS(X2) + CS(T, X1)$ to show how tram dags can model interactions. Maybe just refer in the methods for the CS of tram dags, or alternatively in the ITE with tram dags (where i used CI, but to show that CS is also possible to allow for specific interactions)

Complex shift (Interaction example) to show what is also possible:

Here I just want to make a short input from another example. So there the true model was that of a logistic regression with the binary outcome Y and 3 predictors. The binary treatment T and the two continuous predictors X1 and X2. There was also an interaction effect assumed between treatment and X1. So this basically means that the effect of X1 on the outcome is different for the two treatment groups.
And here we can show that our TRAM-DAG specified by a complex shift of T and X1 can also capture this interaction effect quite well.


\section{Negative Log Likelihood}


\subsection{Continuous Outcome}

% Emphasized core components
For a continuous outcome Y the CDF is given by:

\begin{equation}
F_{Y \mid \mathbf{X} = \mathbf{x}}(y) = F_Z(h(s(y) \mid \mathbf{x}))
\end{equation}

where in our case \( F_Z \) is the cumulative distribution function of the standard logistic distribution

\begin{equation}
F_Z(z) = \frac{1}{1 + e^{-z}}, \quad z \in \mathbb{R}
\end{equation}

and \( h \) is the conditional transformation function that maps the scaled outcome \( s(y) \) to the latent scale Z (log-odds).

The outcome $y$ has to be scaled onto the range $[0, 1]$, because the Bernstein polynomial is bounded:

\begin{equation}
s(y) = \frac{y - \min(y)}{\max(y) - \min(y)}
\end{equation}

This scaling also has to be considered when taking the derivative to get the PDF with the change of variables formula:

\begin{equation}
f_{Y \mid \mathbf{X} = \mathbf{x}}(y) = f_Z(h(s(y) \mid \mathbf{x})) \cdot h'(s(y) \mid \mathbf{x}) \cdot s'(y)
\end{equation}

Where $f_Z$ is the PDF of the standard logistic distribution:

\begin{equation}
f_Z(z) = \frac{e^{z}}{(1 + e^{z})^2}, \quad z \in \mathbb{R}
\end{equation}

Finally, the NLL-contributions are then given by the negative log-densities evaluated at the observations.

\begin{equation}
\text{NLL} = - \log (f_{Y \mid \mathbf{X} = \mathbf{x}}(y))
\end{equation}

The full formula is given by

\begin{align}
\text{NLL} = - \log f_{Y \mid \mathbf{X} = \mathbf{x}}(y)
&= -h(s(y) \mid \mathbf{x}) - 2 \log(1 + \exp(-h(s(y) \mid \mathbf{x}))) \nonumber \\
&\quad + \log h'(s(y) \mid \mathbf{x}) - \log(\max(y) - \min(y))
\end{align}




\subsection{Discrete Outcome}


The for a discrete outcome (binary, ordinal, categoric) with categories $y_k$, $k = 1, \ldots, K$, the CDF is given by:

\begin{equation}
F(Y_k \mid \mathbf{X}) = F_Z(h(y_k \mid \mathbf{x}))
\end{equation}

The likelihood contributions are then given by

\begin{equation}
l_i(y_k \mid \mathbf{x}) = f_{Y_k \mid \mathbf{X} = \mathbf{x}}(y_k) =
    \begin{cases}
      F_Z(h(y_k \mid \mathbf{x})) & k=1\\
      F_Z(h(y_k \mid \mathbf{x})) - F_Z(h(y_{k-1} \mid \mathbf{x})) & k=2,\ldots, K-1\\
      1- F_Z(h(y_{k-1} \mid \mathbf{x})) & k = K
    \end{cases}
\end{equation}


from which the NLL-contributions are derived

\begin{equation}
\text{NLL} = - \log (f_{Y_k \mid \mathbf{X} = \mathbf{x}}(y)
\end{equation}


\section{Interpretation of Linear Coefficients} \label{sec:interpretation_linear_coefficients}

The transformation model framework allows for interpretation of the coefficients in the linear shift. Consider the conditional transformation function:

\begin{equation}
F_{X_2 \mid X_1}(x_2) = \operatorname{expit}( h(x_2) + \beta_{12} x_1 ),
\end{equation}

where \( h(x_2) \) is a smooth, monotonic transformation (e.g., a Bernstein polynomial), and \( \beta_{12} \) is a linear coefficient encoding the effect of \( X_1 \) on \( X_2 \).

Taking the logit (inverse expit) yields:

\begin{equation}
\log\left( \frac{F_{X_2 \mid X_1}(x_2)}{1 - F_{X_2 \mid X_1}(x_2)} \right)
= h(x_2) + \beta_{12} x_1.
\end{equation}

This linear additive structure allows the interpretation of \( \beta_{12} \). The odds ratio when increasing \( x_1 \) by one unit is:

\begin{equation}
\text{OR}_{x_1 \to x_1 + 1} = 
\frac{\exp(h(x_2) + \beta_{12}(x_1 + 1))}{\exp(h(x_2) + \beta_{12} x_1)} 
= \exp(\beta_{12}).
\end{equation}

\noindent\textbf{Interpretation:} The quantity \( \exp(\beta_{12}) \) represents the \textbf{multiplicative change in odds} for \( X_2 \le x_2 \) when increasing \( X_1 \) by one unit, holding all else constant.



\section{Encoding of discrete variables} \label{sec:encoding_discrete_variables}

In the TRAM-DAG a variable $X_i$ can act as a predictor variable for a child node, or as an outcome (child node) that depends on some parent nodes. When $X_i$ is acting as an outcome, the distribution of the variable $X_i$ represented by the transformation function $h$ which estimates a cut-point for each variable. So different form of intercept $h_i$ is used compared to a continuous outcome variable.

If a discrete variable $X_i$ with $K$ categories is used as a predictor variable, it should be dummy encoded. This is done by creating $K-1$ binary variables, where each variable indicates whether the observation belongs to this specific category/level or not. The first category/level is used as the reference and is not explicitly included in the model.

Example: for an ordinal variable $X_i$ with three levels (1, 2 3), we create two binary variables:

\begin{itemize}
  \item $X_{i,1}$: 1 if $X_i = 2$, 0 otherwise
  \item $X_{i,2}$: 1 if $X_i = 3$, 0 otherwise
\end{itemize}

Assume a continuous outcome $Y$ that depends on the ordinal variable $X$ with 3 levels, the CDF for $Y$ is given by: 
$F(Y \mid X=1) = F_Z(h_I(y) + x_1\beta_1 + x_2\beta_2)$ 

For $X=1$, the reference level, the CDF simplifies to: 
$F(Y \mid X=1) = F_Z(h_I(y))$

For $X=2$, the CDF becomes: $F(Y \mid X=1) = F_Z(h_I(y) + \beta_1)$

For $X=3$, the CDF becomes: $F(Y \mid X=1) = F_Z(h_I(y) + \beta_2)$

The coefficients $\beta_1$ and $\beta_2$ can be interpreted as the additive shift in the latent scale $h_I(y)$ when moving from the reference level (1) to levels 2 and 3, respectively.


\section{Scaling of continuous variables} \label{sec:scaling_continuous_variables}

Neural networks work best when the input variables are standardized. A linear, monotonic and invertible transformation of a predictor variable changes the interpretation of the coefficient. Scaling a predictor variable $X$ as $X_{\text{std}} = (X - mean(X)) / sd(X)$ will imply that the coefficient $\tilde{\beta}$ is interpreted as the change in log-odds for a one standard deviation increase in the predictor variable or equivalently, for a one unit increase in the standardized predictor. This is different from the interpretation of the coefficient $\beta$ in the original scale, which represents the change in log-odds for a one unit increase in the predictor variable.




In contrast, the standardization of the outcome variable has no effect on the interpretation (because the scale invariance of the log-odds). Consider, we standardize the outcome \( Y \) as follows:

\[
Y_{\text{std}} = \frac{Y - \mu_Y}{\sigma_Y}
\]

This transformation is linear, monotonic, and invertible:

\[
Y = Y_{\text{std}} \cdot \sigma_Y + \mu_Y
\]

Therefore, for any threshold \( y \), we have the equivalence:

\[
P(Y < y \mid X) = P\left(Y_{\text{std}} < \frac{y - \mu_Y}{\sigma_Y} \mid X\right)
\]

This means that the probability is the identical when evaluating the same quantile in the standardized outcome as in the raw outcome. Furthermore, the interpretation of coefficients in a continuous outcome logistic regression remains unchanged. In particular, the log-odds ratio:

\[
\log \left( \frac{P(Y < y \mid X + 1)}{1 - P(Y < y \mid X + 1)} \right) -
\log \left( \frac{P(Y < y \mid X)}{1 - P(Y < y \mid X)} \right)
\]

is equal to:

\[
\log \left( \frac{P\left(Y_{\text{std}} < \frac{y - \mu_Y}{\sigma_Y} \mid X + 1\right)}{1 - P\left(Y_{\text{std}} < \frac{y - \mu_Y}{\sigma_Y} \mid X + 1\right)} \right) -
\log \left( \frac{P\left(Y_{\text{std}} < \frac{y - \mu_Y}{\sigma_Y} \mid X\right)}{1 - P\left(Y_{\text{std}} < \frac{y - \mu_Y}{\sigma_Y} \mid X\right)} \right)
\]

as long as the same quantile (i.e. probability threshold) is used. Thus, the coefficient \( \beta \) reflects the same change in log-odds for a one-unit increase in the (standardized) predictor, regardless if the outcome is standardized or not. This property is also crucial for the evaluation of the bernstein polynomial, since the outcome has to be scaled on a range between 0 and 1.


The general formula of the transformation model is

\[
P(Y < y \mid X = x) = F_z\left(h(Y) + \beta \cdot X\right)
\]

but the model is fitted with standardized outcome and predictors

\[
P(Y_{\text{std}} < y_{\text{std}} \mid X_{\text{std}} = x_{\text{std}}) = F_z\left(\tilde{h}(Y_{\text{std}}) + \tilde{\beta} \cdot X_{\text{std}}\right)
\]

where $\tilde{h}$ and $\tilde{\beta}$ represent the estimated transformation function and coefficients after standardizing the outcome and predictors.

For example, if we want to know the probability \( P(Y < 20 \mid X = 3) \) with standardized variables, the model is specified as

\[
P\left(\frac{Y - \mu_Y}{\sigma_Y} < \frac{20 - \mu_Y}{\sigma_Y} \,\middle|\, X_{\text{std}} = \frac{3 - \mu_X}{\sigma_X} \right)
= F_z\left(\tilde{h}\left(\frac{20 - \mu_Y}{\sigma_Y}\right) + \tilde{\beta} \cdot \frac{3 - \mu_X}{\sigma_X} \right)
\]


\section{Bernstein Polynomial for Continuous Outcomes}

In deep TRAMs the intercept for continuous variables is a smooth monotonically increasing function that is represented by a Bernstein polynomial of order \( K \) (here the complex intercept case where the Intercept already depends on the predictors $x$, however, the same principle that follows also applies for the simple intercept case):

\begin{equation}
h_I(y \mid \mathbf{x}) = \sum_{k=0}^{K} b_k(\mathbf{x}) \cdot B_k(s(y))
\label{eq:bernstein_intercept}
\end{equation}

where \( B_k(s(y)) \) is the Bernstein basis polynomial of order \( K \) evaluated at the scaled outcome \( s(y) \):


To guarantee that the transformation \( h_I(y \mid \mathbf{x}) \) is monotonically increasing in \( y \), the coefficients \( b_k(\mathbf{x}) \) must form a non-decreasing sequence. This is ensured via a *cumulative softmax* parameterization. Instead of learning \( b_k(\mathbf{x}) \) directly as the outputs of the intercept neural network, we first define unbounded parameters \( \theta_k(\mathbf{x}) \in \mathbb{R} \) and then compute the Bernstein parameters using the cumulative softmax transformation:

\begin{equation}
\tilde{b}_k(\mathbf{x}) = \sum_{j=0}^{k} \frac{\exp(\theta_j(\mathbf{x}))}{\sum_{\ell=0}^{K} \exp(\theta_\ell(\mathbf{x}))}, \quad \text{for } k = 0, \ldots, K.
\end{equation}

This transformation produces a vector \( \tilde{b}_k(\mathbf{x}) \) that is monotonically increasing in \( k \), with values bounded in \( [0, 1] \). It ensures that:

- \( \tilde{b}_0(\mathbf{x}) \leq \tilde{b}_1(\mathbf{x}) \leq \ldots \leq \tilde{b}_K(\mathbf{x}) \),
- The sum of increments is 1,
- The transformation is smooth and differentiable.


The combination of Bernstein polynomials with cumulative softmax-transformed parameters allows flexible, smooth, and strictly monotonic transformations of continuous outcomes, which are essential properties for distribution estimation and generative sampling within the deep TRAM architecture.

\subsection{Scaling and Extrapolation of the Bernstein Polynomial}



% :

% \begin{equation}
% s(y) = \frac{y - \min(y)}{\max(y) - \min(y)}
% \end{equation}
Because the Bernstein polynomial is only defined on the range \( [0, 1] \) the outcome y has to be scaled onto the same range. Furthermore, for the sole purpose of estimating the parameters of the Bernstein polynomial it would be sufficient to finish here. However, one has to be able to also evaluate $h(y \mid \mathbf{x})$ for arbitrary values of y, in particular also the ones that are outside of $(\min(y_{train}), \max(y_{train}))$). This is also crucial for sampling. Therefore we extend the Bernstein polynomial by linearly extrapolating the tails of the polynomial. We do this by constructing inside the 5\% and 95\% quantiles of $y$ by the smooth Bernstein polynomial \ref{eq:bernstein_intercept} and linearly extrapolating the outside this range using the slope of the polynomial at the boundaries. This results in a piecewise-defined function that is differentiable, monotonic, and defined for all real values of \( y \), which is essential for evaluating the model at arbitrary outcomes or for generative sampling.

Tho formalize this, let \( q_{0.05} \) and \( q_{0.95} \) denote the 5\% and 95\% empirical quantiles of the outcome \( y \), computed on the training data. The scaled outcome is defined as

\begin{equation}
s(y) = \frac{y - q_{0.05}}{q_{0.95} - q_{0.05}}.
\end{equation}

This scaling maps the interval \( [q_{0.05}, q_{0.95}] \) to the unit interval \( [0, 1] \), which is the domain of the Bernstein basis polynomials. Let \( h_I(s(y) \mid \mathbf{x}) \) be the original transformation as defined in Equation~\eqref{eq:bernstein_intercept}. The extrapolated transformation \( h^*(y \mid \mathbf{x}) \) is then defined as

\begin{equation}
h^*(y \mid \mathbf{x}) =
\begin{cases}
h_I(0 \mid \mathbf{x}) + h_I'(0 \mid \mathbf{x}) \cdot (s(y) - 0), & \text{if } s(y) < 0 \\
h_I(s(y) \mid \mathbf{x}), & \text{if } 0 \leq s(y) \leq 1 \\
h_I(1 \mid \mathbf{x}) + h_I'(1 \mid \mathbf{x}) \cdot (s(y) - 1), & \text{if } s(y) > 1
\end{cases}
\label{eq:extended_bernstein}
\end{equation}

The function is thus extrapolated beyond the central range using the tangent line at the boundaries. The derivatives \( h_I'(0 \mid \mathbf{x}) \) and \( h_I'(1 \mid \mathbf{x}) \) are computed analytically from the Bernstein basis and the learned coefficients \( b_k(\mathbf{x}) \), and ensure continuous differentiability across the domain (see next subsection).


This construction ensures several desirable mathematical properties. First, the transformation \( \tilde{h}(y \mid \mathbf{x}) \) is globally defined on \( \mathbb{R} \), avoiding undefined regions or discontinuities. Second, it preserves monotonicity due to the use of the cumulative softmax parameterization of the coefficients \( b_k(\mathbf{x}) \), which guarantees that the Bernstein polynomial is strictly increasing. Finally, the piecewise-linear extrapolation ensures the function is continuously differentiable and smooth at the junctions \( s(y) = 0 \) and \( s(y) = 1 \).




\subsection{Analytical Derivative of the Bernstein Polynomial Transformation}

To efficiently compute the gradient of the transformation \( h_I(y \mid \mathbf{x}) \) with respect to its inputs, we can exploit the analytical structure of the Bernstein basis polynomials. Recall the general form of the transformation:

\begin{equation}
h_I(y \mid \mathbf{x}) = \sum_{k=0}^{K} b_k(\mathbf{x}) \cdot B_k(s(y)),
\end{equation}

where \( B_k(s) \) are the Bernstein basis polynomials of order \( K \), and \( b_k(\mathbf{x}) \) are predictor-dependent coefficients. For fixed \( \mathbf{x} \), the derivative with respect to \( y \) is needed, for example, to evaluate the density function when \( h_I \) is used in a generative model.

Let us denote \( s = s(y) \). Using the chain rule, we compute:

\begin{equation}
\frac{d}{dy} h_I(y \mid \mathbf{x}) = \sum_{k=0}^{K} b_k(\mathbf{x}) \cdot \frac{d}{dy} B_k(s) = \sum_{k=0}^{K} b_k(\mathbf{x}) \cdot \frac{dB_k(s)}{ds} \cdot \frac{ds}{dy}.
\end{equation}

The derivative of the scaled outcome \( s(y) = \frac{y - q_{0.05}}{q_{0.95} - q_{0.05}} \) is simply

\begin{equation}
\frac{ds}{dy} = \frac{1}{q_{0.95} - q_{0.05}}.
\end{equation}

The derivative of the Bernstein basis polynomial \( B_{k,K}(s) \) is known and given by:

\begin{equation}
\frac{d}{ds} B_{k,K}(s) = K \left[ B_{k-1,K-1}(s) - B_{k,K-1}(s) \right].
\end{equation}

Therefore, the full derivative is:

\begin{equation}
\frac{d}{dy} h_I(y \mid \mathbf{x}) = \frac{K}{q_{0.95} - q_{0.05}} \sum_{k=0}^{K} b_k(\mathbf{x}) \left[ B_{k-1,K-1}(s) - B_{k,K-1}(s) \right].
\end{equation}

This expression can be evaluated efficiently and is used both in the likelihood computation (e.g., via change-of-variables) and for constructing tail extrapolations with matching slopes.




\section{Calibration plots: Experiment 2} \label{sec:calibrations_experiment2}

Figure \ref{fig:calibration_IST_glm} - \ref{fig:calibration_IST_TRAM_DAG} show the calibration plots in terms of the predicted risks against the the observed proportions for the models applied in experiment 2 (International Stroke Trial (IST)). 
% It becomes apparent, that tuning the random forest model out-of-bag leads to a poor calibration on the training set, but due to better generalization it leads to a better calibration on the test set.

\begin{figure}[htbp]
\centering
\includegraphics[width=0.45\textwidth]{img/results_IST/glm_tlearner_train_calibration_plot.png}
\includegraphics[width=0.45\textwidth]{img/results_IST/glm_tlearner_test_calibration_plot.png}
\caption{Calibration plot for the T-learner logistic regression applied on the International Stroke Trial (IST) in experiment 2. It shows the predicted risks against the the observed proportions of the event. Left: training dataset; Right: test dataset.}
\label{fig:calibration_IST_glm}
\end{figure}


\begin{figure}[htbp]
\centering
\includegraphics[width=0.45\textwidth]{img/results_IST/IST_tuned_rf_tlearner_train_calibration_plot.png}
\includegraphics[width=0.45\textwidth]{img/results_IST/IST_tuned_rf_tlearner_test_calibration_plot.png}
\caption{Calibration plot for the T-learner tuned random forest applied on the International Stroke Trial (IST) in experiment 2. It shows the predicted risks against the the observed proportions of the event. Left: training dataset; Right: test dataset.}
\label{fig:calibration_IST_tuned_rf}
\end{figure}


\begin{figure}[htbp]
\centering
\includegraphics[width=0.45\textwidth]{img/results_IST/IST_TRAM_DAG_slearner_train_calibration_plot.png}
\includegraphics[width=0.45\textwidth]{img/results_IST/IST_TRAM_DAG_slearner_test_calibration_plot.png}
\caption{Calibration plot for the S-learner TRAM-DAG applied on the International Stroke Trial (IST) in experiment 2. It shows the predicted risks against the the observed proportions of the event. Left: training dataset; Right: test dataset.}
\label{fig:calibration_IST_TRAM_DAG}
\end{figure}




\section{Default Random Forest for ITE Estimation} \label{sec:default_rf_ite}

In Section \ref{sec:ite_models} we pointed out the importance of calibration of models when estimating individual treatment effects. In this section we show the results of the default random forest model without tuning for scenario (1), illustrated in Figure \ref{fig:fully_observed_dag_rf_appendix} where all variables are observed and there are strong treatment and interaction effects. The results are shown in Figure \ref{fig:fully_observed_glm_rf}. In the scatterplot of true vs. predicted probabilities for $\text{P}(Y_i = 1 \mid  \mathbf{X_i} = \mathbf{x_i}, T_i = t_i)$ in the train set, it is visible that the model does not predict the probabilities accurately, hence is not well calibrated. This poor calibration also translates to the estimated ITEs. In comparison, the results of the tuned random forest in Figure \ref{fig:fully_tuned_rf_tlearner} show that the model is better calibrated and the estimated ITEs are close to the true ITEs. This illustrates the importance of tuning models for ITE estimation, as poor calibration can lead to biased estimates of individualized treatment effects.


\begin{figure}[htbp]
\centering
\includegraphics[width=0.35\textwidth]{img/results_ITE_simulation/simulation_observed.png}
\caption{DAG for scenario (1), where all variables are observed and there are strong treatment and interaction effects. The numbers indicate the coefficients on the log-odds-scale. Red: interaction effects between treatment ($T$) and covariates ($X_1$ and $X_2$) on the outcome ($Y$).}
\label{fig:fully_observed_dag_rf_appendix}
\end{figure}


\begin{figure}[htbp]
\centering
\includegraphics[width=0.9\textwidth]{img/results_ITE_simulation/fully_observed_rf_tlearner.png}
\caption{Results with the default random forest in scenario (1) when the DAG is fully observed and there are strong treatment and interaction effects. Left: true vs. predicted probabilities for $\text{P}(Y=1 \mid X, T)$; Middle: true vs. predicted ITEs; Right: observed ATE in terms of risk difference per estimated ITE subgroup.}
\label{fig:fully_observed_glm_rf}
\end{figure}




\section{Calibration differences for complex model: Experiment 3} \label{sec:calibration_tuned_rf}

Figure \ref{fig:calibration_tuned_rf} shows the calibration plots in terms of the predicted risks against the the observed proportions of the event for the T-learner tuned random forest for scenario (3) with weak direct and interaction treatment effects. This is in contrast to the prediction plots presented in Section \ref{sec:results_experiment3} where we presented the true probabilities of the event $\text{P}(Y=1 \mid X, T)$ against the predicted probabilities. It becomes apparent, that tuning the random forest model out-of-bag leads to a poor calibration on the training set, but due to better generalization it leads to a better calibration on the test set.

\begin{figure}[htbp]
\centering
\includegraphics[width=0.45\textwidth]{img/results_ITE_simulation/small_interaction_tuned_rf_tlearnertrain_calibration_plot.png}
\includegraphics[width=0.45\textwidth]{img/results_ITE_simulation/small_interaction_tuned_rf_tlearnertest_calibration_plot.png}
\caption{Calibration plot for the T-learner tuned random forest for scenario (3) with weak direct and interaction treatmetn effects. It shows the predicted risks against the the observed proportions of the event. Left: training dataset; Right: test dataset.}
\label{fig:calibration_tuned_rf}
\end{figure}




%%%%%%%%%%%%%%%%%%%%%%%%%%%%%%%%%%%%%%%%%%%%%%%%%%%%%%%%%%%%%%%%%%%%%%
%%%%%%%%%%%%%%%%%%%%%%%%%%%%%%%%%%%%%%%%%%%%%%%%%%%%%%%%%%%%%%%%%%%%%%


\cleardoublepage

\end{document}



%%% Local Variables:
%%% ispell-local-dictionary: "en_US"
%%% End:
