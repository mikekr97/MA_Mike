% LaTeX file for Chapter 05


\chapter{Conclusions}

% Benefit of TRAM-DAGS: our model "knows when to adjust" due to the causal graph and learned functions. With classical regression approaches, there is the danger of wrongly adjusting for covariates and therefore obtaining misleading parameter estimates for inference. A tram dag is a generative causal model because once fitted it can be used to generate samples from the distributions (including interventional and counterfactual.). Hybrid modelling (interpretable, flexible, etc) is a unique capability of this model, to our knowledge. " chatgpt: combining the robustness of structural causal modeling with the representational power of modern neural networks. Furthermore, the ability to selectively impose interpretability on certain components makes the model suitable for real-world tasks requiring both transparency and flexibility."

%  for interpretation see pearl book 2009 p. 366. the key is to say leaving all other variables "untouched" and not "constant". he also talks about the connection to the do-operator.


The poor performance in the IST dataset was likely due to true weak heterogeneity or due to unobserved variables. We come to this conclusion because of our simulations in experiment 3 that revealed these possible problems.


We showed how TRAM-DAGS can be applied do estimate the causal relationships in a given fully observed DAG. We pointed out the importance of individualized treatment effects, for example in personnalized medicine or targeted marketing. Calibration of causal ML models is key to achieve an accurate ITE estimation. Also the trade off between complexity and generalizability becomes more important in this application compared to sole predictive modelling. We pointed out potential pitfalls that can emerge in real world settings and should be paid attention towards. These can be for example too little heterogeneity or general poor effect of the treatment, or the fact that there could be unobserved effect modifiers (treatment-covariate interactions). In terms of effect modifiers, methods in literature have already been proposed such as instrumental variables (IV) or Negative Controls (?) where additional variables in a special dependency to the treatment and exposure are used to adjust for unobserved variables (confounders or effect modifiers?). However, it strongly depends on the setting and it is not guaranteed that there exist such supporting variables. We claim that if we know the structure of the DAG, with TRAM-DAGs we can estimate the ITE regardless if we have a RCT or observational data. The only requirement is that the DAG is correct and fully observed, i.e. no unobserved confounders or effect modifiers exist. And since the average treatment effect (ATE) is the average of the individual treatment effects (ITE), we can also estimate the ATE from the ITEs. This implies that running an expensive RCT is not necessary if we have a good observational dataset and know the DAG structure. Our last experiment supports this claim. We used the medians of the potential outcomes to calculate the ITE, however, if the ITE was calculated based on the expected values, it would be directly comparable to the ATE from the RCT in terms of the difference in means, which might be a more classical measure. 
