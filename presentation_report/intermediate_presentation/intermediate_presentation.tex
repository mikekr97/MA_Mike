% \documentclass[aspectratio=169,onlytextwidth,english]{beamer}
\documentclass[onlytextwidth,english]{beamer}\usepackage[]{graphicx}\usepackage[]{xcolor}
% maxwidth is the original width if it is less than linewidth
% otherwise use linewidth (to make sure the graphics do not exceed the margin)
\makeatletter
\def\maxwidth{ %
  \ifdim\Gin@nat@width>\linewidth
    \linewidth
  \else
    \Gin@nat@width
  \fi
}
\makeatother

\definecolor{fgcolor}{rgb}{0.345, 0.345, 0.345}
\newcommand{\hlnum}[1]{\textcolor[rgb]{0.686,0.059,0.569}{#1}}%
\newcommand{\hlsng}[1]{\textcolor[rgb]{0.192,0.494,0.8}{#1}}%
\newcommand{\hlcom}[1]{\textcolor[rgb]{0.678,0.584,0.686}{\textit{#1}}}%
\newcommand{\hlopt}[1]{\textcolor[rgb]{0,0,0}{#1}}%
\newcommand{\hldef}[1]{\textcolor[rgb]{0.345,0.345,0.345}{#1}}%
\newcommand{\hlkwa}[1]{\textcolor[rgb]{0.161,0.373,0.58}{\textbf{#1}}}%
\newcommand{\hlkwb}[1]{\textcolor[rgb]{0.69,0.353,0.396}{#1}}%
\newcommand{\hlkwc}[1]{\textcolor[rgb]{0.333,0.667,0.333}{#1}}%
\newcommand{\hlkwd}[1]{\textcolor[rgb]{0.737,0.353,0.396}{\textbf{#1}}}%
\let\hlipl\hlkwb

\usepackage{framed}
\makeatletter
\newenvironment{kframe}{%
 \def\at@end@of@kframe{}%
 \ifinner\ifhmode%
  \def\at@end@of@kframe{\end{minipage}}%
  \begin{minipage}{\columnwidth}%
 \fi\fi%
 \def\FrameCommand##1{\hskip\@totalleftmargin \hskip-\fboxsep
 \colorbox{shadecolor}{##1}\hskip-\fboxsep
     % There is no \\@totalrightmargin, so:
     \hskip-\linewidth \hskip-\@totalleftmargin \hskip\columnwidth}%
 \MakeFramed {\advance\hsize-\width
   \@totalleftmargin\z@ \linewidth\hsize
   \@setminipage}}%
 {\par\unskip\endMakeFramed%
 \at@end@of@kframe}
\makeatother

\definecolor{shadecolor}{rgb}{.97, .97, .97}
\definecolor{messagecolor}{rgb}{0, 0, 0}
\definecolor{warningcolor}{rgb}{1, 0, 1}
\definecolor{errorcolor}{rgb}{1, 0, 0}
\newenvironment{knitrout}{}{} % an empty environment to be redefined in TeX

\usepackage{alltt}

% use official beamer theme from uzh
\usetheme[english]{uzh} 

% First installation of languages required
% tinytex::tlmgr_install("babel-english")
% tinytex::tlmgr_install("babel-german")


%% load relevant packages:


\usepackage[T1]{fontenc}
\usepackage[latin9]{inputenc}
%\usepackage[english]{babel}
\usepackage{pgfpages}           % necessary for the handouts production
\usepackage{amsmath}            % for nice mathematics
\usepackage{verbatim}           % for verbatim output
\usepackage{wasysym}            % symbols (smilies etc.)
\usepackage{longtable}
\usepackage{float}
\usepackage{textcomp}
\usepackage{graphicx}
\usepackage{xcolor} % for the color names, see: http://en.wikibooks.org/wiki/LaTeX/Colors#Predefined_
\usepackage{natbib}             % for bibliography style and citations
\usepackage{hyperref}
\usepackage{caption}
\hypersetup{%
    hyperindex=true,
    colorlinks=true,%
    urlcolor = {uzh@blue},% in theme uzh
    citecolor = {uzh@blue},
    urlcolor = {uzh@berry},
    pdfstartview=Fit,%
    pdfpagelayout=SinglePage,%
    pdfpagemode=UseThumbs
  }%
\usepackage{url}
\DeclareOptionBeamer{compress}{\beamer@compresstrue}
\ProcessOptionsBeamer

%% define slidetitle color
\setbeamercolor{title}{fg=uzh@blue}
\setbeamercolor{frametitle}{fg=uzh@blue}


\title{Neural Causal Models with TRAM-DAGs}

%% The following are all optional, simply comment them
%\subtitle{Subtitle (optional)}
\institute{Master Program in Biostatistics www.biostat.uzh.ch\\ Master Thesis: Intermediate Presentation}  %% optional
% leave some vertical space here

\author{Mike Kr{\"a}henb{\"u}hl, Supervisors: Beate Sick, Oliver D{\"u}rr }
\date{\today}
\titlegraphic{img/uzh-lake.jpg}


%%%%%%%%%%%%%%%%%%%%%%%%%%%%%%%%%%%%%%%%%%%%%%%%%%%%%%%% 


\IfFileExists{upquote.sty}{\usepackage{upquote}}{}
\begin{document}

\maketitle




\begin{frame}{Background}

\begin{columns}

% Left side: Text (approx. 3/4 of the slide)
\begin{column}{0.7\textwidth}
\textbf{Supervisors:}
\begin{itemize}
    \item Beate Sick, UZH
    \item Oliver D{\"u}rr, HTWG Konstanz
\end{itemize}

\textbf{Paper \textit{"Interpretable Neural Causal Models with TRAM-DAGs"} \citep{sick2025}:}
\begin{itemize}
    \item Framework to model causal relationships
    \item Based on transformation models
    \item Rely on (deep) neural networks
    \item Compromise between interpretability and flexibility
\end{itemize}
\end{column}

% Right side: Image (approx. 1/4 of the slide)
\begin{column}{0.3\textwidth}
\includegraphics[width=\textwidth]{img/TRAM_DAG_Background.png}
\end{column}

\end{columns}

\end{frame}




\begin{frame}{Research Questions}

\citet{sick2025} showed on synthetic data, that TRAM-DAGs can be fitted on observational data and tackle causal queries on all three levels of Pearl's causal hierarchy.

\textbf{In my thesis:}


\begin{enumerate}
    \item Apply the framework on real-world data
    
    \begin{itemize}
        \item DAG has to be defined
        \item Ground-truth is not known
    \end{itemize}
    
    \item Individualized Treatment Effect estimation
    \begin{itemize}
        \item Potential outcomes under different treatments
        \item Crucial for personalized medicine
    \end{itemize}
\end{enumerate}
\end{frame}



% \begin{frame}{Table of Contents}
% 
% 
% \begin{itemize}
%     \item Review on causality and TRAMs
%     \item TRAM-DAGs
%     \item Example on a simulation
%     \item Next steps
% \end{itemize}
% \end{frame}




\begin{frame}{RCT vs. Observational Data}

\vspace{1cm}

\begin{columns}

% Left side: Text
\begin{column}{0.43\textwidth}
\textbf{Randomized Controlled Trial:}
\begin{itemize}
    \item Gold standard for estimating causal effect
    \item Solves problem of confounding
\end{itemize}

\end{column}

% \hspace{0.5cm}

\begin{column}{0.45\textwidth}
\textbf{Observational Data:}
\begin{itemize}
    \item Real world, potential confounding
    \item We assume no unobserved confounding
\end{itemize}
\end{column}

\end{columns}


% Below: image
\includegraphics[width=\textwidth]{img/RCT_Observational.png}


\end{frame}

% 
% \begin{frame}{Causality: Example}
%     \centering
%     \vspace{0.2cm}
% <<dag_smoking, echo=FALSE, fig.width=3, fig.height=3, out.width="40%">>=
% theme_set(theme_dag())
% 
% # Set custom coordinates
% coord_dag <- list(
%   x = c(smoking = 0, age = 1, exercise = 2, heart = 2),
%   y = c(smoking = 1.3, age = 1.5, exercise = 1.5, heart = 1.3)
% )
% 
% smoking_ca_dag <- dagify(heart ~ smoking + age + exercise,
%   smoking ~ age,
%   labels = c(
%     "heart" = "Heart Disease",
%     "smoking" = "Smoking",
%     "exercise" = "Exercise",
%     "age" = "Age"
%   ),
%   #latent = "unhealthy",
%   exposure = "smoking",
%   outcome = "heart",
%   coords = coord_dag
% )
% 
% # plot with reduced size
% ggdag(smoking_ca_dag, text = FALSE, use_labels = "label") + theme_void()
% 
% 
% # Use label repel to avoid overlap with arrows
% ggdag(smoking_ca_dag, text = TRUE) +
% 	geom_dag_label(aes(label = label, fill = FALSE), 
%                   label.size = 0.2,
%                   size = 3.5, 
%                   color = "black", 
%                   segment.color = "black", 
%                   segment.size = 0.2) +
%   theme_void()
% 
% @
% \end{frame}
\begin{frame}{Structural Causal Model}

\textbf{SCM:} Describes the causal mechanism and probabilistic uncertainty

\vspace{0.2cm}

\begin{itemize}
    \item $X_i$ = observed variable
    \item $U_i$ = noise distribution
    % \item $f$ = in our case: $X_2 = f(X_1, U) = h^{-1}(U_{\text{logis}} - \mathbf{x}^\top \boldsymbol{\beta})$
\end{itemize}

\vfill

\centering
\includegraphics[width=1\textwidth]{img/SCM.png}

\end{frame}






\begin{frame}{Estimating Functional Form}

% Statistical Methods
\begin{columns}
\begin{column}{0.75\textwidth}
\textbf{Statistical methods:}
\begin{itemize}
    \item E.g. linear/logistic regression
    \item Predefined form, risk of bias if misspecified
\end{itemize}
\end{column}
\begin{column}{0.23\textwidth}
\includegraphics[width=\textwidth]{img/conditional_distributions.png}
\end{column}
\end{columns}

\vspace{0.3cm}

% Neural Networks
\begin{columns}
\begin{column}{0.75\textwidth}
\textbf{Neural networks:}
\begin{itemize}
    \item E.g. feed-forward NNs, normalizing flows, VACAs
    \item Flexible, but "black-box", data-type limitations
\end{itemize}
\end{column}
\begin{column}{0.23\textwidth}
\includegraphics[width=\textwidth]{img/neural_network.png}
\end{column}
\end{columns}

\vspace{0.3cm}

% TRAM-DAGs
\begin{columns}
\begin{column}{0.75\textwidth}
\textbf{TRAM-DAGs:}
\begin{itemize}
    \item Compromise: flexibility + interpretability
    \item Mixed data types
\end{itemize}
\end{column}
\begin{column}{0.23\textwidth}
\includegraphics[width=\textwidth]{img/TRAM_Raw.png}
\end{column}
\end{columns}

\end{frame}







\begin{frame}{Pearl's Causality Ladder}

\begin{columns}

% Left side: Text (60%)
\begin{column}{0.60\textwidth}

\textbf{Observational (seeing)} \\
$P(Y=1 \mid E=1)$ \\
{\footnotesize \textit{"Probability of heart disease given that the person exercises"}}

\vspace{0.4cm}

\textbf{Interventional (doing)} \\
$P(Y=1 \mid \text{do}(E=1))$ \\
{\footnotesize \textit{"Probability of heart disease if we made people start exercising"}} 

\vspace{0.4cm}

\textbf{Counterfactual (imagining)} \\
 $P(Y_{(E=1)} = 1 \mid E=0, Y=1)$ \\
{\footnotesize \textit{"Would someone who does not exercise and has heart disease still have it if they had exercised?"}}

\end{column}

% Right side: Image (40%)
\begin{column}{0.40\textwidth}
\includegraphics[width=1\linewidth]{img/Pearls_Ladder.png}
\end{column}

\end{columns}

\end{frame}


\begin{frame}{Individualized Treatment Effect (ITE)}

Difference in outcomes between two treatment options, for one specific individual with unique characteristics.
    \[
    \text{ITE}_\text{i} = \text{P}(\text{Y}_\text{i} = 1 \mid \text{T} = 1, \mathbf{X} = \mathbf{x_i}) - \text{P}(\text{Y}_\text{i} = 1 \mid \text{T} = 0, \mathbf{X} = \mathbf{x_i})
    \]
    
\textbf{Difficulty:} \\
\begin{itemize}
	% \item Personalized medicine
	\item We can only observe one facutal outcome - the other one is counterfactual
    % \item \textbf{Average treatment effect:}
    % \[
    % \text{ATE} = \text{E}\left[\text{ITE}_i \right]
    % \]
\end{itemize}


\begin{columns}

% Left side: Text
\begin{column}{0.55\textwidth}

\textbf{Recent findings:} \\
\begin{itemize}
    \item \citet{chen2025} analyzed mainstream causal ML methods for ITE estimation on two large RCTs.
    \item ITEs estimated
from training data failed to generalize to the test data
 \end{itemize}
\end{column}

% Right side: Image
\begin{column}{0.45\textwidth}
\includegraphics[width=\textwidth]{img/ATE_ITE.png}
\end{column}
\end{columns}

\end{frame}





\begin{frame}{Transformation Models}

Flexible distributional regression method \citep{hothorn2014}

\vspace{0.4cm}

\textbf{Continuous } $Y \in \mathbb{R}$: 
\[
F_{Y \mid \mathbf{X} = \mathbf{x}}(y) = F_Z(h(y) + \mathbf{x}^\top \boldsymbol{\beta})
\]

\textbf{Discrete } $Y \in \{y_1, y_2, \ldots, y_K\}$: 
\[
P(Y \leq y_k \mid \mathbf{X} = \mathbf{x}) = F_Z(\vartheta_k + \mathbf{x}^\top \boldsymbol{\beta}), \quad k = 1, 2, \ldots, K - 1
\]

\vspace{0.4cm}

\begin{itemize}
    \item $F_Z$: CDF of the standard logistic distribution
    \item $h$: Transformation function, monotonically increasing
    \item $\mathbf{x}$: Predictors
\end{itemize}

\end{frame}






\begin{frame}{Transformation Models}

\begin{columns}

% Left column: Continuous Y
\begin{column}{0.48\textwidth}
\textbf{Continuous $Y$:}

{\small
\vspace{0.2cm}
Intercept: Bernstein polynomial
\vspace{0.2cm}

\scalebox{0.85}{$
h_I(y) = \frac{1}{M + 1} \sum_{k=0}^{M} \vartheta_k \, \text{B}_{k, M}(y)
$}

\vspace{0.2cm}

\scalebox{0.85}{$
h(y \mid \mathbf{x}) = h_I(y) - \mathbf{x}^\top \boldsymbol{\beta}
$}
}

\end{column}

% Right column: Discrete/Ordinal Y
\begin{column}{0.48\textwidth}
\textbf{Discrete/Ordinal $Y$:}

{\small

\vspace{0.2cm}
Intercept: Cut-off value
\vspace{0.2cm}

\scalebox{0.85}{$
h_I(y_k) = \vartheta_k
$}

\vspace{0.2cm}

\scalebox{0.85}{$
h(y_k \mid \mathbf{x}) = h_I(y_k) - \mathbf{x}^\top \boldsymbol{\beta}
$}
}

\end{column}

\end{columns}

\vspace{0.3cm}
\centering
\includegraphics[width=0.9\textwidth]{img/TRAM_Cont_Ord.png}

\end{frame}








\begin{frame}{Deep TRAMs}
  \begin{itemize}
    \item Extended to Deep TRAMs \citep{sick2020}
    \item Flexible components
    \item Minimize the NLL through NN optimization
  \end{itemize}

  \vfill
  \centering
  \includegraphics[width=0.9\linewidth]{img/deep_TRAM.png}
\end{frame}





\begin{frame}{TRAM-DAGs}

  \centering
  \includegraphics[width=1\linewidth]{img/TRAM_DAG.png}
\end{frame}



\begin{frame}{Simulation Example}
  \begin{itemize}
    \item We have:
    \begin{itemize}
      \item Observational data (simulated)
      \item Predefined DAG
    \end{itemize}
    \item We want:
    \begin{itemize}
      \item Estimate conditional CDF of each variable
      \item Sample from conditional distributions for causal queries
    \end{itemize}
  \end{itemize}

  \vfill
  \centering
  \includegraphics[width=0.7\linewidth]{img/Simulation_Example.png}
\end{frame}




\begin{frame}{Adjacency Matrix}

Model structure represented by a meta-adjacency matrix:

\begin{itemize}
  \item \textbf{Rows}: source of effect
  \item \textbf{Columns}: target of effect
\end{itemize}

\vspace{0.4cm}

\begin{center}
\begin{tikzpicture}[baseline={(current bounding box.center)}]

  % DAG image
  \node (img) at (0, 0) {\includegraphics[width=0.28\textwidth]{img/DAG_MA.png}};

  % Matrix
  \node (matrix) at (5, 0) {
    $\mathbf{MA} =
    \begin{bmatrix}
      0 & \text{LS} & \text{LS} \\
      0 & 0  & \text{CS} \\
      0 & 0  & 0
    \end{bmatrix}$
  };

  % Arrow
  \draw[->, thick] (img.east) -- (matrix.west);

\end{tikzpicture}
\end{center}

\end{frame}



\begin{frame}{Data Generating Process (DGP)}

\begin{columns}

% Left column: Descriptions and formulas
\begin{column}{0.72\textwidth}

\textbf{\(X_1\):} Continuous, bimodal. \textit{Source node} (independent).

\vspace{0.4cm}

\textbf{\(X_2\):} Continuous. Depends on \(X_1\) (\textcolor{red}{linear}):

\vspace{0.15cm}
{\scriptsize
\[
\textcolor{red}{\beta_{12} = 2}, \quad h_I(X_2) = 5 X_2
\]
\[
\boxed{
h(X_2 \mid X_1) = h_I(X_2) + \textcolor{red}{\beta_{12}} X_1
}
\]
}

\vspace{0.4cm}

\textbf{\(X_3\):} Ordinal. Depends on \(X_1\) (\textcolor{red}{linear}) and \(X_2\) (\textcolor{blue}{complex}):

\vspace{0.15cm}
{\scriptsize
\[
\textcolor{red}{\beta_{13} = 0.2}, \quad \textcolor{blue}{f(X_2) = 0.5 \cdot \exp(X_2)}, \quad \vartheta_k \in \{-2,\, 0.42,\, 1.02\}
\]
\[
\boxed{
h(X_{3,k} \mid X_1, X_2) = \vartheta_k + \textcolor{red}{\beta_{13}} X_1 + \textcolor{blue}{f(X_2)}
}
\]
}

\end{column}

% Right column: Plot
\begin{column}{0.28\textwidth}
\includegraphics[width=0.8\linewidth]{img/DGP_Variables.png}
\end{column}

\end{columns}

\end{frame}





% \textbf{Parameters:} 281 total (not all used)
% \begin{itemize}
%     \item Simple Intercepts (SI): \textbf{240} \\
%     - only 43 needed (20 + 20 + 3)
%     \item Linear Shifts (LS): \textbf{9} \\
%     - 2 active
%     \item Complex Shifts (CS): \textbf{32} \\
%     - 24 active
% \end{itemize}

% 
% \begin{frame}{Construct Model: Modular Neural Network}
% 
% \begin{columns}
% 
% % Left side: Text
% \begin{column}{0.65\textwidth}
% 
% \vspace{0.1cm}
% 
% \textbf{Inputs:} \\ Observations + adjacency matrix
% 
% \vspace{0.4cm}
% 
% \textbf{Outputs:}
% \begin{itemize}
%     \item Simple Intercepts (SI): \vartheta
%     \item Linear Shifts (LS): $\beta_{12}X_1, \beta_{13}X_2$
%     \item Complex Shift (CS):  $\beta(X_2)$
% \end{itemize}
% 
% 
% 
% \vspace{0.4cm}
% \textbf{Transformation Functions:} \\
% \begin{align*}
% h(X \mid pa(X)) = \text{SI} + \text{LS} + \text{CS} \\
% & h(X_1) = h_I(X_1) \\
% & h(X_2 \mid X_1) = h_I(X_2) + \textcolor{red}{\beta_{12} X_1} \\
% & h(X_{3,k} \mid X_1, X_2) = \vartheta_k + \textcolor{red}{\beta_{13} X_1} + \textcolor{blue}{\beta(X_2)} 
% % \quad \( h = \text{SI} + \text{LS} + \text{CS} \)
% \end{align*}
% \end{column}
% 
% % Right side: Image
% \begin{column}{0.35\textwidth}
% \begin{figure}
%   \centering
%   \includegraphics[width=0.65\linewidth]{img/CS.png}
%   \caption{$\text{CS}_{X_2}$ on $X_3$}
% \end{figure}
% 
% \end{column}
% 
% \end{columns}
% 
% \end{frame}
% 


\begin{frame}{Construct Model: Modular Neural Network}

\begin{columns}

% Left side: Text
\begin{column}{0.65\textwidth}

\vspace{0.1cm}

\textbf{Inputs:} \\ Observations + adjacency matrix

\vspace{0.4cm}

\textbf{Outputs:}
\begin{itemize}
    \item Simple Intercepts (SI): $\textcolor{violet}{\vartheta}$
    \item Linear Shifts (LS): $\textcolor{red}{\beta_{12}X_1}, \textcolor{red}{\beta_{13}X_2}$
    \item Complex Shift (CS):  $\textcolor{blue}{\beta(X_2)}$
\end{itemize}

\vspace{0.4cm}
\textbf{Transformation Functions:}
\begin{align*}
& \boxed{h(X_i \mid pa(X_i)) = \text{SI} + \text{LS} + \text{CS}} \\
& h(X_1) = \textcolor{violet}{h_I(X_1)} \\
& h(X_2 \mid X_1) = \textcolor{violet}{h_I(X_2)} + \textcolor{red}{\beta_{12} X_1} \\
& h(X_{3,k} \mid X_1, X_2) = \textcolor{violet}{\vartheta_k} + \textcolor{red}{\beta_{13} X_1} + \textcolor{blue}{\beta(X_2)} 
\end{align*}

\end{column}

% Right side: Image
\begin{column}{0.35\textwidth}
\begin{figure}
  \centering
  \includegraphics[width=0.65\linewidth]{img/CS.png}
  \caption{$\text{CS}_{X_2}$ on $X_3$}
\end{figure}
\end{column}

\end{columns}

\end{frame}


% 
% \begin{frame}{The Setting}
% 
% \begin{columns}
% 
% % Left column: What we have
% \begin{column}{0.48\textwidth}
% \textbf{What we have:}
% \begin{itemize}
%     \item Data
%     \item DAG (expert knowledge or structure discovery)
%     \item No hidden confounding
%     \item Adjacency matrix
% \end{itemize}
% \end{column}
% 
% % Right column: What we want
% \begin{column}{0.48\textwidth}
% \textbf{What we want:}
% \begin{itemize}
%     \item Model each variable as a function of its parents
%     \item Use transformation models: $F_Z(h(y \mid x))$
%     \item Perform causal inference:
%     \begin{itemize}
%         \item Sample from observational distribution
%         \item Sample from interventional distribution
%         \item Answer counterfactual queries
%     \end{itemize}
% \end{itemize}
% \end{column}
% 
% \end{columns}
% 
% \end{frame}



% Simulation experiment from the file:
% experiment_5_ordinal_outcome.R

% 
% \begin{frame}{Loss: Negative Log-Likelihood (NLL)}
% 
% % Emphasized core components
% \textbf{CDF, density and NLL of the TRAM (for continuous outcome):}
% \[
% F_{Y \mid \mathbf{X} = \mathbf{x}}(y) = F_Z (h(s(y) \mid \mathbf{x}))
% \]
% 
% \[
% f_{Y \mid \mathbf{X} = \mathbf{x}}(y) = f_Z (h(s(y) \mid \mathbf{x})\right) \cdot h'\left(s(y) \mid \mathbf{x}\right) \cdot s'(y)
% \]
% 
% \[
% \begin{aligned}
% \text{NLL} = - \log f_{Y \mid \mathbf{X} = \mathbf{x}}(y)
% &= -h(s(y) \mid \mathbf{x}) - 2 \log\left(1 + \exp\left(-h(s(y) \mid \mathbf{x})\right)\right) \\
% &\quad + \log h'(s(y) \mid \mathbf{x}) - \log(\max(y) - \min(y))
% \end{aligned}
% \]
% 
% \vspace{0.4cm}
% 
% % Supporting details in smaller columns
% \begin{columns}
% \begin{column}{0.5\textwidth}
% {\scriptsize
% \textbf{Standard Logistic Density:}
% \[
% f_Z(z) = \frac{e^{z}}{(1 + e^{z})^2}, \quad z \in \mathbb{R}
% \]
% }
% \end{column}
% 
% \begin{column}{0.5\textwidth}
% {\scriptsize
% \textbf{Scaled $y$ (Bernstein Polynomial on $[0,1]$):}
% \[
% s(y) = \frac{y - \min(y)}{\max(y) - \min(y)}
% \]
% }
% \end{column}
% \end{columns}
% 
% \end{frame}




\begin{frame}{Loss: Negative Log-Likelihood (NLL)}

% Emphasized core components
\textbf{CDF, density and NLL of the TRAM (for continuous outcome):}
\[
F_{Y \mid \mathbf{X} = \mathbf{x}}(y) = F_Z(h(s(y) \mid \mathbf{x}))
\]

\[
f_{Y \mid \mathbf{X} = \mathbf{x}}(y) = f_Z(h(s(y) \mid \mathbf{x})) \cdot h'(s(y) \mid \mathbf{x}) \cdot s'(y)
\]


\[
\text{NLL} = - \log (f_{Y \mid \mathbf{X} = \mathbf{x}}(y))
\]

%  full NLL
% \[
% \begin{aligned}
% \text{NLL} = - \log f_{Y \mid \mathbf{X} = \mathbf{x}}(y)
% &= -h(s(y) \mid \mathbf{x}) - 2 \log(1 + \exp(-h(s(y) \mid \mathbf{x}))) \\
% &\quad + \log h'(s(y) \mid \mathbf{x}) - \log(\max(y) - \min(y))
% \end{aligned}
% \]

\vspace{0.4cm}

% Supporting details in smaller columns
\begin{columns}
\begin{column}{0.5\textwidth}
{\scriptsize
\textbf{Standard logistic density:}
\[
f_Z(z) = \frac{e^{z}}{(1 + e^{z})^2}, \quad z \in \mathbb{R}
\]
}
\end{column}

\begin{column}{0.5\textwidth}
{\scriptsize
\textbf{Scaled $y$ (Bernstein polynomial bounded $[0,1]$):}
\[
s(y) = \frac{y - \min(y)}{\max(y) - \min(y)}
\]
}
\end{column}
\end{columns}

\end{frame}




\begin{frame}{Model Fitting}
  \begin{columns}
    \begin{column}{0.5\linewidth}
      \begin{itemize}
        \item Samples: 20'000 training / 5'000 validation
        \item Learning rate: 0.005
        \item Epochs: 400
      \end{itemize}
    \end{column}
    \begin{column}{0.5\linewidth}
      \centering
      \includegraphics[width=\linewidth]{img/Loss_Example.png}
    \end{column}
  \end{columns}
\end{frame}






\begin{frame}{Interpretable Coefficients}

\begin{columns}

% Left column: Smaller formulas
\begin{column}{0.42\textwidth}

\textbf{Linear Shift Coefficients:}

\vspace{0.2cm}

{\scriptsize
\[
F(x_2 \mid x_1) = F_Z\left(h(x_2) - x_1 \textcolor{red}{\beta_{12}}\right)
\]


\[
F(x_3 \mid x_1, x_2) = F_Z\left(h(x_3) - x_1 \textcolor{red}{\beta_{13}} - CS_{x_2}\right)
\]
}

\end{column}

% Right column: Bigger plot
\begin{column}{0.58\textwidth}

\includegraphics[width=\linewidth]{img/Betas.png}

\end{column}

\end{columns}

\end{frame}






\begin{frame}{Interpretable Coefficients}
\small
\[
F_{X_2 \mid X_1}(x_2) = \operatorname{expit}( h(x_2) + \beta_{12} x_1)
\]

\vspace{0.3cm}

\[
\log\left( \frac{F_{X_2 \mid X_1}(x_2)}{1 - F_{X_2 \mid x_1}(x_2)} \right)
= \text{expit}^{-1}( \text{expit}(h(x_2) + \beta_{12} x_1 ))
= h(x_2) + \beta_{12} x_1
\]

\vspace{0.3cm}

\[
\begin{aligned}
\text{OR}_{x_1 \to x_1 + 1}
&= \frac{\text{odds}(X_2 \le x_2 \mid X_1 = x_1 + 1)}{\text{odds}(X_2 \le x_2 \mid X_1 = x_1)} = \frac{\exp(h(x_2) + \beta_{12}(x_1 + 1))}{\exp(h(x_2) + \beta_{12} x_1)} = \exp(\beta_{12})
\end{aligned}
\]

\vspace{0.4cm}

\textbf{Interpretation:} \\
\(\exp(\beta_{12})\) is the \textbf{multiplicative change in odds} for \(X_2 \le x_2\) when increasing \(\mathbf{X}_1\) by 1 unit, \emph{holding all else constant}.

\end{frame}







\begin{frame}{Linear and Complex Shifts}


\includegraphics[width=\linewidth]{img/LS_CS.png}


\end{frame}






\begin{frame}{Intercepts}


\includegraphics[width=\linewidth]{img/baseline_trafo.png}


\end{frame}



\begin{frame}{What is Possible with a Fitted TRAM-DAG?}

Once the model is fitted, it can be used:

\begin{itemize}
\item as genearative sampling model to sample observational and interventional distributions
\item or to determine counterfactual outcomes in the continuous case.
\end{itemize}


Even if the model is fitted on observational data, we can make interventional and counterfactual statements, given the DAG is correct and no unobserved confounding.


\end{frame}



\begin{frame}{Sampling from the Fitted TRAM-DAG (observational)}

\begin{columns}

% Left column: Sampling explanation
\begin{column}{0.65\textwidth}

\textbf{Nodes $X_i , i \in \{1,\, 2,\, 3\}$:}

\vspace{0.2cm}

\begin{itemize}
    \item Sample latent value: 
    \[
    z_i \sim F_{Z_i} \quad \text{(e.g., \texttt{rlogis()} in R)}
    \]

    \item Determine \(x_i\) such that:

    \begin{itemize}
        \item \textbf{If \(X_i\) is continuous:}
        \[
        x_i = h^{-1}(z_i \mid \text{pa}(x_i))
        \]
        
        \item \textbf{If \(X_i\) is ordinal:}
        find the smallest category $x_i$ such that
        \[
        x_i = \min \left\{ x : z_i \le h(x \mid \text{pa}(x_i)) \right\}
        \]
    \end{itemize}
\end{itemize}

\end{column}

% Right column: Illustration
\begin{column}{0.3\textwidth}
\includegraphics[width=0.9\linewidth]{img/Sampling.png}
\end{column}

\end{columns}

\end{frame}




\begin{frame}{Sampling from the Fitted TRAM-DAG (interventional)}

\textbf{Interventional sampling:} \\

\begin{itemize}
    \item Do-intervention: \( \textcolor{red}{\text{do}(x_2 = \alpha})\)
    \item Sample from the interventional-distribution:
\end{itemize}
\[
x_3 = \min \left\{ x : z_3 \le h(x \mid x_1, \textcolor{red}{x_2 = \alpha}) \right\}
\]

\centering
\includegraphics[width=0.5\linewidth]{img/interventional.png}


\end{frame}


\begin{frame}{Sampling Distributions}


\begin{columns}

\begin{column}{0.75\linewidth}
    \includegraphics[width=1\linewidth]{img/Sampling_Distributions.png}
  \end{column}
  
  
  \begin{column}{0.25\linewidth}
    \includegraphics[width=1\linewidth]{img/Do_sampling.png}
  \end{column}
  
\end{columns}



\end{frame}




%  new slide for the image Prob_Estimates.png

\begin{frame}{Observational and Interventional Queries}

\includegraphics[width=1\linewidth]{img/Prob_Estimates.png}

\end{frame}




\begin{frame}{Example: ITE Estimation}


\centering
\includegraphics[width=0.3\linewidth]{img/ITE_DAG.png}

% \begin{columns}
% 
% % Left column: formulas + two images
% \begin{column}{0.75\textwidth}
\scriptsize

\vspace{0.2cm}
\textbf{DGP:} \quad
$\text{logit}(P(Y=1 \mid T, X_1, X_2)) = \beta_0 + X_1 \beta_1 + X_2 \beta_2 + T \beta_3 + \textcolor{red}{T X_1\beta_4}$

\vspace{0.2cm}

\textbf{TRAM-DAG:} \quad
$h(Y_k \mid T, X_1, X_2) = \vartheta_k + \textcolor{red}{\text{CS}(T, X_1)} + \text{LS}(X_2)$

% \end{column}

% Right column: small DAG image
% \begin{column}{0.25\textwidth}
% \vspace{-0.2cm}
% \centering
% \includegraphics[width=1\linewidth]{img/ITE_DAG.png}
% \end{column}

% \end{columns}

\vspace{-0.1cm}

\begin{columns}

% Small left image
\begin{column}{0.48\textwidth}
\begin{figure}
  \centering
  \includegraphics[width=0.7\linewidth]{img/NN_CS_T_X1.png}
  \caption{$\text{CS}_{(T, X_1)}$ on $Y$}
\end{figure}
\end{column}

% Wide right image
\begin{column}{0.48\textwidth}
\centering
\includegraphics[width=\linewidth]{img/CS_T_X1.png}
\end{column}

\end{columns}



\end{frame}


\begin{frame}{Example: ITE Estimation}

\begin{columns}
% Left column: text
\begin{column}{0.5\textwidth}
\textbf{ITE\textsubscript{i} estimation from fitted model:}

\begin{itemize}
	\item \(\text{P}(\text{Y}_\text{i} = 1 \mid \textcolor{red}{\text{do(T = 1)}}, \mathbf{x_i})\)
	\item \(\text{P}(\text{Y}_\text{i} = 1 \mid \textcolor{red}{\text{do(T = 0)}}, \mathbf{x_i})\)
    \item Calculate ITE\textsubscript{i}
\end{itemize}

\end{column}

\begin{column}{0.5\textwidth}

% two images vertically
\centering
\vspace{0.8cm}
\includegraphics[width=0.7\linewidth]{img/ITE_recal.png}
\vspace{0.8cm}
\includegraphics[width=0.7\linewidth]{img/ATE_ITE_recal.png}
\end{column}
\end{columns}

\end{frame}



% \begin{frame}{Background: The NISCI Study}




% \end{frame}




% \begin{frame}{Background: The NISCI Study}
%     \centering
%     \vspace{0.2cm}
% <<figure_uems_tissue, echo=FALSE>>=
% 
% # quantiles <- quantile(subset(datc, visit_id==2)$SR, prob = 1:3/4, na.rm = TRUE)
% # datc$SRc <- cut(datc$SR, breaks=c(-Inf, quantiles, Inf), as.ordered=TRUE)
% 
% # xyplot(uems ~ tm | trtplot , data = datc, group = id, xlim = c(-5, 240), type = "b", 
% #        between = list(x = 1, y = 1), pch = 20, cex = .7,
% #        lwd = 3, col = rgb(.1, .1, .1, .3), 
% #        xlab = "Time (in days)",
% # 			 ylab = "Upper Extremity Motor Score",
% # 			 main = "Recovery in Upper Extremity Motor Score over time",
% # 			 scales = list(
% #          x = list(alternating = 1), # Show x-axis tick labels only at the bottom
% #          tck = c(1, 0) # Remove top ticks
% #        ),
% #        )
% 
% 
% @
% \end{frame}
% 


% 
% \begin{frame}{Methods: Generalized Additive Model (GAM)}
% 
% % Generalized additive models replace the linear form 
% % $\sum \beta_j X_j$ by a sum of smooth functions $\sum f_j(X_j)$ \citep{hastie1986}.
% 
% %add vertical space
% \vspace{5pt}
% 
% \begin{equation*}
% \text{UEMS}_{ij} = \beta_0 + f_1(\text{tissue}_{i}) + f_2(\text{time}_{ij}) + f_3(\text{tissue}_{i}, \text{time}_{ij}) + b_{0i} + b_{1i} \cdot \text{time}_{ij} + \epsilon_{ij}
% \end{equation*}
% 
% \textbf{Variables:} 
% 
% \begin{itemize}
% \item UEMS: for individual $i$ at time $j$
% \item Tissue: Parasagittal tissue bridges at baseline
% \item Time: standardized (Time=0 corresponds to day of first medication and Time=1 is at 6-months follow up)
% \item Random slope and intercept for each individual $i$
% \item \(\epsilon_{ij} \sim \mathcal{N}(0, \sigma^2)\)
% \end{itemize}
% 
% 
% 
% \end{frame}



% \begin{frame}{Results: Baseline Tissue Bridges}
%     \centering
%     \vspace{0.2cm}
% <<figure_baseline_uems_tissue, echo=FALSE, error=FALSE, warning=FALSE, results='asis'>>=
% # 
% # xyplot(uems ~ SR | trtplot, 
% #        data = subset(datc, visit_id == 2), 
% #        group = id, 
% #        type = "b", 
% #        pch = 16,
% #        cex = 1.2,
% #        lwd = 2,
% #        col = rgb(.1, .1, .1, .5),
% #        xlab = "Parasagittal tissue bridges",
% #        ylab = "Upper Extremity Motor Score",
% #        main = "Baseline tissue bridges vs UEMS",
% #        grid = TRUE,
% # 			 scales = list(
% #          x = list(alternating = 1), # Show x-axis tick labels only at the bottom
% #          tck = c(1, 0) # Remove top ticks
% #        ),
% # 			 panel = function(x, y, ...) {
% #          panel.xyplot(x, y, ...) # Plot the points
% #          panel.lmline(x, y, col = "black", lwd = 1)  # Add a linear regression line
% #        }
% #        )
% 
% 
% @
% \end{frame}
% 
% 
% 


\begin{frame}{Outlook}


\textbf{What we also did:}

\begin{itemize}
    \item Ordinal predictors
    \item TRAM-DAG on real climate data
\end{itemize}

\vspace{0.4cm}

\textbf{What comes next:}
\begin{itemize}
    \item ITE estimation on real RCT-data
    \item If time: include image data
\end{itemize}


\end{frame}




\begin{frame}{References}
  \small
  \bibliographystyle{apalike}
\bibliography{C:/Users/kraeh/OneDrive/Dokumente/Desktop/UZH_Biostatistik/Masterarbeit/MA_Mike/presentation_report/literature/bibSTA490}
\end{frame}

\end{document}
